%%%%%%%%%%%%%%%%%%%%%%%%%%%%%%%%%%%%%%%%%%%%%%%%%%%%%%%%%%%%%%%%%%%%%%%%%
%%
%W  infra.tex       ANUPQ documentation - infrastructure    Werner Nickel
%W                                                            Greg Gamble
%%
%H  $Id$
%%
%%

%%%%%%%%%%%%%%%%%%%%%%%%%%%%%%%%%%%%%%%%%%%%%%%%%%%%%%%%%%%%%%%%%%%%%%%%%
\Chapter{Infrastructure}

Most of the details in this chapter are of a technical nature;  the  user
need only skim over this chapter on a first reading. Mostly, it is enough
to know that

\beginlist%unordered

\item{$\bullet$} 
you must do a `RequirePackage("anupq");' before you can expect to  use  a
command defined by the {\ANUPQ} package (details are in  Section~"Loading
the ANUPQ Package");

\item{$\bullet$}
partial results of {\ANUPQ} commands and some other data  are  stored  in
the `ANUPQData' global variable (details are  in  Section~"The  ANUPQData
Record"); 

\item{$\bullet$} 
doing `SetInfoLevel(InfoANUPQ, <n>);' for <n> greater  than  the  default
value 1 will give progressively more information  of  what  is  going  on
``behind the scenes'' (details are in Section~"Setting the  Verbosity  of
ANUPQ via Info and InfoANUPQ");

\item{$\bullet$} 
if you passed options to a function and ran into an error then you should
almost certainly type `ResetOptionsStack();' so  those  options  are  not
still on the `OptionsStack' to affect subsequent functions  (details  are
in Section~"Hints and Warnings regarding the use of Options");

\item{$\bullet$} 
in Section~"Utility functions" we describe functions `PqLeftNormComm' for
computing  left  normed  commutators  in  {\GAP},  `PqGAPRelators'  which
converts strings representing relators from  a  format  the  `pq'  program
understands to a form that  {\GAP}  understands,  and  `PqExample'  which
executes examples for the  {\ANUPQ}  package  or  displays  an  index  of
examples; and

\item{$\bullet$} 
in Section~"Attributes and a Property for fp and pc p-groups" we describe
the  attributes  and  property  `NuclearRank',  `MultiplicatorRank'   and
`IsCapable'.

\endlist

%%%%%%%%%%%%%%%%%%%%%%%%%%%%%%%%%%%%%%%%%%%%%%%%%%%%%%%%%%%%%%%%%%%%%%%%%
\Section{Loading the ANUPQ Package}

\index{banner}
To use the {\ANUPQ} package, as with  any  {\GAP}  package,  it  must  be
requested explicitly. This is done by calling

\beginexample
gap> RequirePackage("anupq");
#I    Loading the ANUPQ (ANU p-Quotient) package
#I    C code by  Eamonn O'Brien <obrien@math.auckland.ac.nz>
#I                ANU pq binary version: 1.5
#I    GAP code by Werner Nickel <nickel@mathematik.tu-darmstadt.de>
#I            and   Greg Gamble  <gregg@math.rwth-aachen.de>
#I                ANUPQ package version: 1.1
#I  
#I                For help, type: ?ANUPQ
true
\endexample

The `RequirePackage' command is described in Section~"ref:RequirePackage"
in the {\GAP} Reference Manual.

If {\GAP} cannot find a working `pq' binary, the call to `RequirePackage'
will return `fail'.

If you know you have a working `pq'  program,  as  well  as  a  correctly
installed {\ANUPQ}  package,  it  is  possible  to  suppress  the  `Info'
messages by temporarily setting the `InfoLevel' of  `InfoWarning'  to  0,
and a duplicated semicolon will suppress the `true' result:

\beginexample
gap> SetInfoLevel(InfoWarning, 0); RequirePackage( "anupq" );;
gap> SetInfoLevel(InfoWarning, 1);
\endexample

\index{banner!suppression}
The banner is also suppressed if the global {\GAP}  variable  `QUIET'  is
`true' or `BANNER' is  `false'  (these  conditions  occur  if  {\GAP}  is
invoked with the `-q' or `-b' command line  switches,  respectively).  If
you want to load the  {\ANUPQ}  package  by  default,  you  can  put  the
`RequirePackage' command into your `.gaprc'  file  (see  Section~"ref:The
.gaprc file" in the {\GAP} Reference Manual).

%%%%%%%%%%%%%%%%%%%%%%%%%%%%%%%%%%%%%%%%%%%%%%%%%%%%%%%%%%%%%%%%%%%%%%%%%
\Section{The ANUPQData Record}

This section contains fairly technical details which may be skipped on an
initial reading.

\>`ANUPQData' V

is a {\GAP} record in which the essential data for  an  {\ANUPQ}  session
within {\GAP} is stored; its fields are:

\beginitems

\quad`binary' & the path of the `pq' binary;

\quad`tmpdir' & the path of the temporary  directory  used  by  the  `pq'
binary and {\GAP} (i.e.~the directory in which all the  `pq''s  temporary
files are created) (also see "ANUPQDirectoryTemporary" below);

\quad`outfile'& the full path of the default `pq' output  file;

\quad`SPimages'& the full path of the file  `GAP_library'  to  which  the
`pq' program writes its Standard Presentation images;

\quad`version'& the version of the current `pq' binary;

\quad`ni' & a data record used by non-interactive  functions  (see  below
and Chapter~"Non-interactive ANUPQ Functions"); and

\quad`io' & list of data records for `PqStart' (see below  and~"PqStart")
processes;

\enditems

Each time an interactive {\ANUPQ}  process  is  initiated  via  `PqStart'
(see~"PqStart"), an identifying number <ioIndex>  is  generated  for  the
interactive process and a record `ANUPQData.io[<ioIndex>]' with  some  or
all of the fields listed below is  created.  Whenever  a  non-interactive
function is called (see Chapter~"Non-interactive ANUPQ  Functions"),  the
record `ANUPQData.ni' is updated with fields that, if bound, have exactly
the same purpose as for a `ANUPQData.io[<ioIndex>]' record. For technical
reasons, prime io indices are avoided, i.e.~`ANUPQData.io' is in  general
*not* a dense list.

\beginitems

\quad`stream'& the  IOStream  opened  for  interactive  {\ANUPQ}  process
<ioIndex> or non-interactive {\ANUPQ} function;

\quad`group'& the group given  as  first  argument  to  `PqStart',  `Pq',
`PqEpimorphism',  `PqDescendants'  or  `PqStandardPresentation'  (or  any
synonymous methods);

\quad`gens'& a list of the generators of the  group  `group'  as  strings
(the same as those passed to the `pq' program);

\quad`rels'& a list of the relators of the group `group' as strings  (the
same as those passed to the `pq' program);

\quad`name'& the name of the group whose pc presentation is defined by  a
call to the `pq' program (according to the `pq' program -- unless you  have
used the `GroupName' option  (see  e.g.~"Pq")  or  applied  the  function
`SetName'  (see~"ref:SetName")  to  the  group,  the   ``generic''   name
`"\<grp>"' is set as a default);

\quad`class'& the largest lower exponent-$p$ central class of a  quotient
group of the group (usually `group') found by a call to the `pq' program;

\quad`forder'& the factored order of the quotient group of largest  lower
exponent-$p$ central class found for the group  (usually  `group')  by  a
call to the `pq' program (this factored order is given as  a  list  `[$p$,
$n$]', indicating an order of $p^n$);

\quad`pcoverclass'&  the  lower  exponent-$p$  central   class   of   the
$p$-covering group of a $p$-quotient of the group (usually `group') found
by a call to the `pq' program;

\quad`workspace'& the workspace set for the `pq' process (either given as
a second argument to `PqStart', or set by default to 10000000);

\quad`menu'& the current menu of the `pq' process  (the  `pq'  program  is
managed by various  menus,  the  details  of  which  the  user  shouldn't
normally need to know about -- the `menu' field remembers which menu  the
`pq' process is currently ``in'');

\quad`outfname' & is the file to which `pq' output is directed, which  is
always `ANUPQData.outfile', except when option `SetupFile' is used with a
non-interactive  function,  in  which   case   `outfname'   is   set   to
`"PQ_OUTPUT"';

\quad`pQuotient' & is set to the value returned by `Pq'  (see~"Pq")  (the
field `pQepi' is also set at the same time);

\quad`pQepi'  &  is  set  to  the  value  returned   by   `PqEpimorphism'
(see~"PqEpimorphism") (the field `pQuotient' is  also  set  at  the  same
time);

\quad`pCover'  &  is  set   to   the   value   returned   by   `PqPCover'
(see~"PqPCover");

\quad`SP' & is set to the value returned by  `PqStandardPresentation'  or
`StandardPresentation'  (see~"PqStandardPresentation!interactive")   when
called interactively, for process <i> (the field `SPepi' is also  set  at
the same time); and

\quad`SPepi'    &    is    set    to    the     value     returned     by
`EpimorphismPqStandardPresentation' or  `EpimorphismStandardPresentation'
(see~"EpimorphismPqStandardPresentation!interactive")     when     called
interactively, for process <i> (the field `SP' is also set  at  the  same
time).

\enditems

\>ANUPQDirectoryTemporary( <dir> ) F

calls the UNIX command `mkdir' to create <dir>, which must be  a  string,
and if successful a directory  object  for  <dir>  is  both  assigned  to
`ANUPQData.tmpdir' and returned. The field  `ANUPQData.outfile'  is  also
set to be a file in `ANUPQData.tmpdir', and on exit from {\GAP} <dir>  is
removed. Most users will never need  this  command;  by  default,  {\GAP}
typically   chooses   a   ``random''   subdirectory   of    `/tmp'    for
`ANUPQData.tmpdir' which may occasionally have  limits  on  what  may  be
written there. `ANUPQDirectoryTemporary' permits the  user  to  choose  a
directory (object) where one is not so limited.

%%%%%%%%%%%%%%%%%%%%%%%%%%%%%%%%%%%%%%%%%%%%%%%%%%%%%%%%%%%%%%%%%%%%%%%%%
\Section{Setting the Verbosity of ANUPQ via Info and InfoANUPQ}

\>`InfoANUPQ' V

The input to and the output from the `pq'  program  is,  by  default,  not
displayed. However the user may choose to  see  some,  or  all,  of  this
input/output.   This   is   done   via   the   `Info'   mechanism    (see
Chapter~"ref:Info Functions" in the {\GAP} Reference  Manual).  For  this
purpose, there is the <InfoClass>  `InfoANUPQ'.  If  the  `InfoLevel'  of
`InfoANUPQ' is high enough each line of `pq' input/output is directed  to
a call to `Info' and will be displayed for the user to see.  By  default,
the `InfoLevel' of `InfoANUPQ' is 1, and it is recommended that you leave
it at this level, or higher. Messages that  the  user  should  presumably
want to see and output from the `pq' program influenced by  the  value  of
the option `OutputLevel' (see the options listed in Section~"Pq"),  other
than timing and memory usage are directed to `Info' at `InfoANUPQ'  level
1.

To turn off *all* `InfoANUPQ' messaging, set the `InfoANUPQ' level to 0.

There are five other user-intended `InfoANUPQ' levels: 2, 3, 4, 5 and 6.

\beginexample
gap> SetInfoLevel(InfoANUPQ, 2);
\endexample

enables the display of most timing and memory usage data  from  the  `pq'
program. (Some timing and memory usage data, particularly when profuse in
quantity, is `Info'-ed at `InfoANUPQ' level 3 instead.)

\beginexample
gap> SetInfoLevel(InfoANUPQ, 3);
\endexample

enables the display of output of the nature of the first two  `InfoANUPQ'
that was not directly invoked by the  user  (e.g.~some  commands  require
{\GAP} to discover something about the current state known  to  the  `pq'
program). In some cases, the  `pq'  program  produces  a  lot  of  output
despite the fact that the  `OutputLevel'  (see~"option  OutputLevel")  is
unset or is set to 0; such output except is also `Info'-ed at `InfoANUPQ'
level 3.

\beginexample
gap> SetInfoLevel(InfoANUPQ, 4);
\endexample

enables the display of all the commands  directed  to  the  `pq'  program,
behind a ```ToPQ> ''' prompt (so that you can  distinguish  it  from  the
output from the `pq' program).

\beginexample
gap> SetInfoLevel(InfoANUPQ, 5);
\endexample

enables the display of the  `pq'  program's  banner,  prompts  for  input.
Finally,

\beginexample
gap> SetInfoLevel(InfoANUPQ, 6);
\endexample

enables the display of all other output from the `pq' program, namely  the
banner and menus. However, the timing data printed when the  `pq'  program
exits can never be observed.

%%%%%%%%%%%%%%%%%%%%%%%%%%%%%%%%%%%%%%%%%%%%%%%%%%%%%%%%%%%%%%%%%%%%%%%%%
\Section{Hints and Warnings regarding the use of Options}

\index{menu item!of pq program}
\index{option!of pq program is a menu item}
*Note:*
By ``options'' we refer to {\GAP} options. The `pq' program also uses  the
term ``option''; to distinguish the two usages  of  ``option'',  in  this
manual we use the term *menu item* to  refer  to  what  the  `pq'  program
refers to as an ``option''.

Options are passed to the {\ANUPQ} interface functions in either  of  the
two usual mechanisms provided by {\GAP}, namely:

\beginlist%unordered

\item{--} options may be set globally using  the  function  `PushOptions'
(see Chapter~"ref:Options Stack" in the {\GAP} Reference Manual); or

\item{--} options may be appended to the argument list  of  any  function
call,   separated   by   a   colon   from   the   argument   list    (see
Chapter~"ref:Function Calls" in the {\GAP} Reference  Manual),  in  which
case they are then passed on recursively to any subsequent inner function
call, which may in turn have options of their own.

\endlist

Particularly,  when  one  is   using   the   interactive   functions   of
Chapter~"Interactive ANUPQ Functions",  one  should,  in  general,  avoid
using the global method of passing options. In fact,  it  is  recommended
that  prior  to  calling  `PqStart'  the  `OptionsStack'  be  empty.  The
essential problem  with  setting  options  globally  using  the  function
`PushOptions' is that options pushed onto `OptionsStack',  in  this  way,
remain there until an explicit `PopOptions()' call is made.

In contrast, options passed in the usual way behind a colon  following  a
function's arguments (see "ref:Function Calls" in  the  {\GAP}  Reference
Manual) are local, and disappear from `OptionsStack' after  the  function
has executed successfully. However, if the function  does  *not*  execute
successfully, i.e.~it runs into error and the user `quit's the  resulting
`break' loop (see Section~"ref:Break  loops"  in  the  Reference  Manual)
rather than attempting to repair the problem and  typing  `return;'  then
the options of that function are *not* cleared  from  `OptionsStack'.  In
such cases, the user will generally want to:

\begintt
gap> ResetOptionsStack();
\endtt

(see Chapter~"ref:ResetOptionsStack"  in  the  {\GAP}  Reference  Manual)
which recursively calls `PopOptions()' until `OptionsStack' is empty.

Nevertheless, a function, that is passed options after  the  colon,  will
also see any global options or any options passed down  recursively  from
functions calling that function, unless those options are over-ridden  by
options passed via the function. Note that duplication  of  option  names
for different programs may lead to  misinterpretations,  and  mis-spelled
options will not be ``seen''.

The   non-interactive   functions   of   Chapter~"Non-interactive   ANUPQ
Functions" that have `Pq' somewhere in their name provide an  alternative
method  of  passing  options  as  additional  arguments.  This  has   the
advantages that options can be abbreviated and mis-spelled  options  will
be trapped.

%%%%%%%%%%%%%%%%%%%%%%%%%%%%%%%%%%%%%%%%%%%%%%%%%%%%%%%%%%%%%%%%%%%%%%%%%
\Section{Utility Functions}

\>PqLeftNormComm( <elts> ) F

returns for a list of elements of some group (e.g.~<elts> may be  a  list
of words in the generators of  a  free  or  fp  group)  the  left  normed
commutator of <elts>, e.g.~if <w1>, <w2>, <w3>  are  such  elements  then
`PqLeftNormComm( [<w1>, <w2>, <w3>] );' is  equivalent  to  `Comm(  Comm(
<w1>, <w2> ), <w3> );'.

\>PqGAPRelators( <group>, <rels> ) F

returns a list of words that {\GAP} understands, given a list  <rels>  of
strings in the string representations of the generators of the fp  or  pc
group <group> prepared as a list of relators for the `pq' program.

*Note:*
The `pq' program does not use `/' to indicate multiplication by an inverse
and uses square brackets to represent (left  normed)  commutators.  Also,
even though the `pq' program accepts relations,  all  elements  of  <rels>
*must* be in relator form, i.e.~a relation of form `<w1> = <w2>' must  be
written as `<w1>*(<w2>)^-1'.

Here is an example:

\beginexample
gap> F := FreeGroup("a", "b");
gap> PqGAPRelators(F, [ "a*b^2", "[a,b]^2*a", "([a,b,a,b,b]*a*b)^2*a" ]);
[ a*b^2, a^-1*b^-1*a*b*a^-1*b^-1*a*b*a, b^-1*a^-1*b^-1*a^-1*b*a*b^-1*a*b*a^
    -1*b*a^-1*b^-1*a*b*a*b^-1*a^-1*b^-1*a^-1*b*a*b^-1*a*b^-1*a^-1*b*a^-1*b^
    -1*a*b*a*b*a^-1*b*a*b^-1*a*b*a^-1*b*a^-1*b^-1*a*b*a*b^-1*a^-1*b^-1*a^
    -1*b*a*b^-1*a*b^-1*a^-1*b*a^-1*b^-1*a*b*a*b^2*a*b*a ]
\endexample

\>PqExample() F
\>PqExample( <example>[, PqStart][, Display] ) F
\>PqExample( <example>[, PqStart][, <filename>] ) F

With no arguments,  or  with  single  argument  `"index"',  or  a  string
<example> that is not the name of a file in the `gap/examples' directory,
an index of available examples is displayed.

With just the one argument <example> that is the name of a  file  in  the
`gap/examples' directory, the example contained in that file is  executed
in its simplest form. Some examples accept options which you may  use  to
modify some of the options used in the commands of the example.  To  find
out which options an example  accepts, use  one  of  the  mechanisms  for
displaying the example described below.

Some examples have both non-interactive and interactive forms; those that
are non-interactive only have a name ending  in  `-ni';  those  that  are
interactive only have a name ending in `-i'; examples with  names  ending
in  `.g'  also  have  only  one  form;  all  other  examples  have   both
non-interactive and interactive forms and for these giving  `PqStart'  as
second argument invokes `PqStart' initially  and  makes  the  appropriate
adjustments  so  that  the  example  is  executed  or   displayed   using
interactive functions.

If `PqExample' is called with last (second or third)  argument  `Display'
then the example  is  displayed  without  being  executed.  If  the  last
argument is a non-empty  string  <filename>  then  the  example  is  also
displayed without being executed but is also written to a file with  that
name. Passing an empty string as last argument has  the  same  effect  as
passing `Display'.

*Note:*
The  variables  used  in  `PqExample'  are  local  to  the   running   of
`PqExample', so there's no  danger  of  having  some  of  your  variables
over-written. However, they are not  completely  lost  either.  They  are
saved to a record `ANUPQData.examples.vars', i.e.~if `F'  is  a  variable
used in the example then you will be able to access it after  `PqExample'
has finished as `ANUPQData.examples.vars.F'.

\>AllPqExamples() F

returns  a  list  of  all  currently  available   examples   in   default
UNIX-listing (i.e.~alphabetic) order.

%%%%%%%%%%%%%%%%%%%%%%%%%%%%%%%%%%%%%%%%%%%%%%%%%%%%%%%%%%%%%%%%%%%%%%%%%
\Section{Attributes and a Property for fp and pc p-groups}

\>NuclearRank( <G> ) A
\>MultiplicatorRank( <G> ) A
\>IsCapable( <G> ) P

return the nuclear rank  of  <G>,  $p$-multiplicator  rank  of  <G>,  and
whether <G> is capable (i.e.~`true' if it is, or `false' if it  is  not),
respectively, where <G> is either an fp $p$-group or a pc $p$-group.

These attributes and property are set automatically if <G> is one of  the
following:

\beginlist%unordered

\item{--}  an  fp   group   returned   by   `PqStandardPresentation'   or
`StandardPresentation' (see~"PqStandardPresentation");

\item{--} the  image  (fp  group)  of  the  epimorphism  returned  by  an
`EpimorphismPqStandardPresentation' or  `EpimorphismStandardPresentation'
call (see~"EpimorphismPqStandardPresentation"); or

\item{--} one of the pc groups of the list  of  descendants  returned  by
`PqDescendants' (see~"PqDescendants").

\endlist

If <G> is not one of the above and the  attribute  or  property  has  not
otherwise been set for <G>, then `PqStandardPresentation'  is  called  to
set all three  of  `NuclearRank',  `MultiplicatorRank'  and  `IsCapable',
before returning the value of the attribute or property actually  called.
Such a group <G> must know in advance that it is a $p$-group; this is the
case for the groups returned by the functions `Pq'  and  `PqPCover',  and
the image group of the epimorphism returned by `PqEpimorphism'.

*Note:* For <G> such that `HasNuclearRank(<G>) = true',  `IsCapable(<G>)'
is equivalent to (the truth or falsity of) `NuclearRank( <G> ) = 0'.

%%%%%%%%%%%%%%%%%%%%%%%%%%%%%%%%%%%%%%%%%%%%%%%%%%%%%%%%%%%%%%%%%%%%%%%%%
%%
%E
