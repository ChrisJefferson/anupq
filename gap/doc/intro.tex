%%%%%%%%%%%%%%%%%%%%%%%%%%%%%%%%%%%%%%%%%%%%%%%%%%%%%%%%%%%%%%%%%%%%%%%%%
\Chapter{Introduction}

The {\GAP} 4 share  package ANU PQ is an interface to  the ANU pq program
making the  functionality of the standalone program  available to {\GAP}.
The ANU pq standalone provides access to implementations of the following
algorithms:

1.  A  $p$-quotient algorithm  to compute  power-commutator presentations
for  p-groups  as  factor  groups  of  finitely  presented  groups.   The
algorithm  implemented here  is based  on  that described  in Newman  and
O'Brien (1996),  Havas and Newman  (1980), and papers referred  to there.
Another description  of the algorithm  appears in Vaughan-Lee  (1990).  A
FORTRAN  implementation of this  algorithm was  programmed by  Alford and
Havas.  The basic data structures of that implementation are retained.

2. A   $p$-group  generation   algorithm  to   generate  power-commutator
presentations of groups of  prime power order.  The algorithm implemented
here is  based on the algorithms  described in Newman  (1977) and O'Brien
(1990).  A FORTRAN implementation of this algorithm was earlier developed
by Newman and O'Brien.

3.   A  standard presentation   algorithm  used  to compute  a  canonical
power-commutator presentation  of a $p$-group. The  algorithm implemented
here is described in O'Brien (1994).

4. An algorithm which can be used  to compute the automorphism group of a
$p$-group. The algorithm implemented here is described in O'Brien (1994).
This part of the standalone  program is not accessible through this share
package.   Instead, users  are advised  to  consider the  {\GAP} 4  share
package AUTPGRP,  which imlements a better algorithm  for the computation
of automorphism groups of $p$-groups.

A reader interested  in details  of the  algorithms  and explanations  of
terms  used  is  referred  to  \cite{NOBr96}, \cite{HN80},  \cite{OBr90},  
\cite{OBr94}, \cite{OBr95},    \cite{New77},    \cite{Vau84},     
\cite{Vau90a},  and \cite{Vau90b}.

For  details about  the implementation and the standalone version see the
file README. This implementation was developed in C by

\begintt
Eamonn O'Brien
Department of Mathematics
University of Auckland
Private Bag 92019
Auckland
New Zealand

e-mail: obrien@math.auckland.ac.nz 
\endtt

The {\GAP} 4 version of this package was adapted from the {\GAP} 3
version by  
\begintt
Werner Nickel
AG 2, Fachbereich Mathematik, TU Darmstadt
Schlossgartenstr. 7, 64289 Darmstadt, Germany

email: nickel@mathematik.tu-darmstadt.de
\endtt
