%%%%%%%%%%%%%%%%%%%%%%%%%%%%%%%%%%%%%%%%%%%%%%%%%%%%%%%%%%%%%%%%%%%%%%%%%
%%
%W  intro.tex       ANUPQ documentation - introduction      Werner Nickel
%%
%%
%H  $Id$
%%
%%

%%%%%%%%%%%%%%%%%%%%%%%%%%%%%%%%%%%%%%%%%%%%%%%%%%%%%%%%%%%%%%%%%%%%%%%%%
\Chapter{Introduction}

\index{ANUPQ}
The {\GAP} 4 share package {\ANUPQ} provides an interface to the ANU PQ C
progam `pq' written by Eamonn O'Brien, making the functionality of the  C
program available to {\GAP}. Henceforth, we shall refer to  the  {\ANUPQ}
interface (resp. the `pq' binary), when discussing the  {\GAP}  component
(resp. the C code component) of the  {\ANUPQ}  share  package.  The  `pq'
program consists of implementations of the following algorithms:

\beginlist

\item{1.}
A *$p$-quotient algorithm* to  compute  pc-presentations  for  $p$-factor
groups of finitely presented groups.

%The algorithm implemented here is based on that described by  Newman  and
%O'Brien \cite{NO96}, Havas and Newman~\cite{HN80}, and papers referred to
%there. Another description of  the  algorithm  is  given  by  Vaughan-Lee
%(see~\cite{Vau90a}, \cite{Vau90b}).  A  FORTRAN  implementation  of  this
%algorithm was programmed by Alford and Havas. The basic  data  structures
%of that implementation are retained.

\item{2.} 
A *$p$-group generation algorithm* to generate pc-presentations of groups
of prime power order.

%The algorithm implemented here is based on the  algorithms  described  by
%Newman~\cite{New77} and O'Brien~\cite{OBr90}. A FORTRAN implementation of
%this algorithm was developed earlier by Newman and O'Brien.

\item{3.}
A  *standard  presentation  algorithm*  used  to  compute   a   canonical
pc-presentation of a $p$-group.

%The algorithm implemented here is described in~\cite{OBr94}.

\item{4.} 
An algorithm which can be used to compute the *automorphism group*  of  a
$p$-group.

\item{}
This part of the `pq' program is  not  accessible  through  the  {\ANUPQ}
share package. Instead, users are advised to consider the {\GAP}~4  share
package {\AutPGrp}, which implements a better algorithm in {\GAP} for the
computation of automorphism groups of $p$-groups.
%The algorithm implemented here is described in~\cite{OBr94}. 

\endlist

%Further   background   may   be   found   in~\cite{OBr95},   \cite{Vau84}
%and~\cite{NNN98}.

The {\ANUPQ} Share Package  Manual  has  been  written  for  {\GAP}  4.3.
Nevertheless, the {\ANUPQ} Share Package is compatible with  {\GAP}  4.2,
but since it uses the iostream technology introduced in  {\GAP}  4.2,  it
requires at least {\GAP} 4.2.

A comprehensive description of the background  and  terminology  used  is
given in Chapter~"Mathematical Background and  Terminology",  which  also
cites a number of references for further reading. Section~"Authors" gives
contact details for the authors of the components of the  {\ANUPQ}  share
package.

Instructions on how to install the {\ANUPQ} share package are located  in
Chapter~"Installing the ANUPQ Share  Package"  and  Section~"Loading  the
ANUPQ Share Package" describes how to load the {\ANUPQ} share package.

Section~"The ANUPQData Record" describes the global variable  `ANUPQData'
in which various data including the temporary directory in which  various
input and output files are created for and by the `pq' binary and various
results of functions are retained. On an initial reading you can probably
skip this section.

Section~"Setting the Verbosity of ANUPQ via Info and InfoANUPQ" describes
an `Info' class which when set to appropriate levels allows one to see if
desired all the dialogue that passes between {\GAP} and the `pq' binary.

Most {\ANUPQ} functions  use  options  and  Section~"Hints  and  Warnings
regarding the use of Options" gives some fairly important hints regarding
these. Please take the time to read it.

Chapter~"Non-interactive   ANUPQ    functions"    describes    all    the
non-interactive functions  of  the  {\ANUPQ}  share  package;  these  are
``one-shot'' functions that invoke the `pq' binary in  such  a  way  that
once {\GAP} has got what it needs, the `pq' is allowed  to  exit.  It  is
expected that most of the time users will only need these functions.

Finally,  Chapter~"Interactive  ANUPQ  functions"   describes   all   the
interactive functions of the {\ANUPQ} share package; these are  functions
that extract information via a dialogue  with  a  running  `pq'  process.
Occasionally, a user needs the ``next step''; the functions  provided  in
this chapter make use of data from previous steps retained  by  the  `pq'
binary, thus allowing the user to interact with the `pq' binary like  one
can when one uses the `pq' binary as a  stand-alone  (see~`guide.dvi'  in
the `standalone-doc' directory).

%%%%%%%%%%%%%%%%%%%%%%%%%%%%%%%%%%%%%%%%%%%%%%%%%%%%%%%%%%%%%%%%%%%%%%%%%
\Section{Authors}

The C implementation of the ANU `pq' standalone was developed by

\begintt
Eamonn O'Brien
Department of Mathematics
University of Auckland
Private Bag 92019
Auckland
New Zealand
\endtt
{\kernttindent}`email:' \Mailto{obrien@math.auckland.ac.nz}

The {\GAP} 4 version of this package was adapted from the {\GAP} 3
version by  

\begintt
Werner Nickel
AG 2, Fachbereich Mathematik, TU Darmstadt
Schlossgartenstr. 7, 64289 Darmstadt, Germany
\endtt
{\kernttindent}`email:' \Mailto{nickel@mathematik.tu-darmstadt.de}

An  interactive  interface  using  iostreams  was  developed   with   the
assistance of Werner Nickel by

%\begintt
\){\kernttindent}Greg Gamble
\){\kernttindent}Lehrstuhl D f\accent127ur Mathematik
\){\kernttindent}Templergraben 64, RWTH Aachen, Germany
%\endtt

{\kernttindent}`email:' \Mailto{gregg@math.rwth-aachen.de}

%The authors would like to thank and acknowledge the careful proof-reading
%and advice provided by Joachim Neub\"user; we would like  to  also  thank
%him for formulating Chapter~"Mathematical Background and Terminology".

%%%%%%%%%%%%%%%%%%%%%%%%%%%%%%%%%%%%%%%%%%%%%%%%%%%%%%%%%%%%%%%%%%%%%%%%%
%%
%E
