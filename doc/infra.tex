%%%%%%%%%%%%%%%%%%%%%%%%%%%%%%%%%%%%%%%%%%%%%%%%%%%%%%%%%%%%%%%%%%%%%%%%%
%%
%W  infra.tex       ANUPQ documentation - infrastructure    Werner Nickel
%W                                                            Greg Gamble
%%
%H  $Id$
%%
%%

%%%%%%%%%%%%%%%%%%%%%%%%%%%%%%%%%%%%%%%%%%%%%%%%%%%%%%%%%%%%%%%%%%%%%%%%%
\Chapter{Infrastructure}

Most of the details in this chapter are of a technical nature;  the  user
need only skim over this chapter on a first reading. Mostly, it is enough
to know that

\beginlist%unordered

\item{$\bullet$} 
you must do a `LoadPackage("anupq");' before you  can  expect  to  use  a
command defined by the {\ANUPQ} package (details are in  Section~"Loading
the ANUPQ Package");

\item{$\bullet$}
partial results of {\ANUPQ} commands and some other data  are  stored  in
the `ANUPQData' global variable (details are  in  Section~"The  ANUPQData
Record"); 

\item{$\bullet$} 
doing `SetInfoLevel(InfoANUPQ, <n>);' for <n> greater  than  the  default
value 1 will give progressively more information  of  what  is  going  on
``behind the scenes'' (details are in Section~"Setting the  Verbosity  of
ANUPQ via Info and InfoANUPQ");

\item{$\bullet$} 
in  Section~"Utility functions"  we describe  some utility  functions and
functions  that run  examples from  the  collection of  examples of  this
package;

\item{$\bullet$} 
in Section~"Attributes and a Property for fp and pc p-groups" we describe
the  attributes  and   property  `NuclearRank',  `MultiplicatorRank'  and
`IsCapable'; and

\item{$\bullet$} 
in Section~"Hints and Warnings regarding the use of Options" we  describe
some troubleshooting strategies. If you are still  using  {\GAP}~4.2,  it
will be important to know to use  `ResetOptionsStack();'  if  you  passed
options to a function and ran into an error. Also this section  describes
how  setting   `ANUPQWarnOfOtherOptions   :=   true;'   may   be   useful
(particularly for  novice  users)  for  detecting  misspelt  options  and
diagnosing other option usage problems.

\endlist

%%%%%%%%%%%%%%%%%%%%%%%%%%%%%%%%%%%%%%%%%%%%%%%%%%%%%%%%%%%%%%%%%%%%%%%%%
\Section{Loading the ANUPQ Package}

\index{banner}
To use the {\ANUPQ} package, as with  any  {\GAP}  package,  it  must  be
requested explicitly. This is done by calling

\beginexample
gap> LoadPackage( "anupq" );
-------------------------------------------------------------
Loading ANUPQ 2.1 (ANU p-Quotient package)
C code by  Eamonn O'Brien <obrien@math.auckland.ac.nz>
           (ANU pq binary version: 1.7)
GAP code by Werner Nickel <nickel@mathematik.tu-darmstadt.de>
        and   Greg Gamble  <gregg@math.rwth-aachen.de>

            For help, type: ?ANUPQ
-------------------------------------------------------------
true
\endexample

\index{banner!suppression}
The banner may be suppressed by providing the version string (`"2.1"') as
second argument and  `false'  as  third  argument  to  the  `LoadPackage'
command.    The     `LoadPackage'     command     is     described     in
Section~"ref:LoadPackage" in the {\GAP} Reference Manual.

Note  that  since  {\ANUPQ}  package  functions  often  need  to  compute
automorphism groups of $p$-groups, and uses the functionality provided by
the {\AutPGrp} package to do so. Therefore, the user should  ensure  that
the {\AutPGrp} package is  also  installed,  preferably  version  1.2  or
better (when present {\AutPGrp} is normally  autoloaded  when  {\GAP}  is
started up). If the {\AutPGrp} package is  not  installed,  the  {\ANUPQ}
package will still load but  an  error  will  arise  the  first  time  an
{\AutPGrp} function is required to compute the automorphism  group  of  a
$p$-group.

If {\GAP} cannot find a working `pq' binary, the  call  to  `LoadPackage'
will return `fail'.

If you want to load the {\ANUPQ} package by  default,  you  can  put  the
`LoadPackage' command  into  your  `.gaprc'  file  (see  Section~"ref:The
.gaprc file" in the {\GAP} Reference Manual). By the way, the novice user
of the {\ANUPQ} package should probably also append the line

\begintt
ANUPQWarnOfOtherOptions := true;
\endtt

to their `.gaprc' file, somewhere after  the  `LoadPackage(  "anupq"  );'
command (see~"ANUPQWarnOfOtherOptions").

*Loading ANUPQ with GAP 4.3*

The remainder of this section is only relevant if  you  are  still  using
{\GAP} 4.3, and note that  in  any  case,  at  least  {\GAP}  4.3fix4  is
required.

To load {\ANUPQ} with {\GAP} 4.3 you will need  to  use  `RequirePackage'
rather than `LoadPackage', and in this case the banner will look a little
different. In particular, each line starts with `\#I', since  the  banner
is `Info'-ed at `InfoWarning' level 1, and so will not be observed if the
`InfoLevel' of `InfoWarning' is 0.

With {\GAP} 4.3, the banner is  also  suppressed  if  the  global  {\GAP}
variable `QUIET' is `true' or `BANNER' is `false' (these conditions occur
if {\GAP} is invoked  with  the  `-q'  or  `-b'  command  line  switches,
respectively).

\atindex{option pkgbanner}{@option \noexpand`pkgbanner'}
With {\GAP} 4.3, there is yet one  more  way  of  suppressing  {\ANUPQ}'s
banner is to supply the option  `pkgbanner'  (which  does  nothing  under
{\GAP} 4.4) with value `"none"' to the `RequirePackage' command:

\beginexample
gap> RequirePackage( "anupq" : pkgbanner := "none" );
true
\endexample

Finally (with {\GAP} 4.3), the full banner is  replaced  by  a  ``short''
1-line loading message (`Info'-ed at `InfoWarning' level 1) if,  instead,
`pkgbanner' has the value `"short"':

\begintt
gap> RequirePackage( "anupq" : pkgbanner := "short" );
#I  Loading ANUPQ Package Version 2.1
true
\endtt

%%%%%%%%%%%%%%%%%%%%%%%%%%%%%%%%%%%%%%%%%%%%%%%%%%%%%%%%%%%%%%%%%%%%%%%%%
\Section{The ANUPQData Record}

This section contains fairly technical details which may be skipped on an
initial reading.

\>`ANUPQData' V

is a {\GAP} record in which the essential data for  an  {\ANUPQ}  session
within {\GAP} is stored; its fields are:

\beginitems

\quad`binary' & the path of the `pq' binary;

\quad`tmpdir' & the path of the temporary  directory  used  by  the  `pq'
binary and {\GAP} (i.e.~the directory in which all the  `pq''s  temporary
files are created) (also see "ANUPQDirectoryTemporary" below);

\quad`outfile'& the full path of the default `pq' output  file;

\quad`SPimages'& the full path of the file  `GAP_library'  to  which  the
`pq' program writes its Standard Presentation images;

\quad`version'& the version of the current `pq' binary;

\quad`ni' & a data record used by non-interactive  functions  (see  below
and Chapter~"Non-interactive ANUPQ Functions");

\quad`io' & list of data records for `PqStart' (see below  and~"PqStart")
processes;

\quad`topqlogfile' & name of file logged to by `ToPQLog' (see~"ToPQLog");
and

\quad`logstream' & stream of file logged to by `ToPQLog' (see~"ToPQLog").

\enditems

Each time an interactive {\ANUPQ}  process  is  initiated  via  `PqStart'
(see~"PqStart"), an identifying number <ioIndex>  is  generated  for  the
interactive process and a record `ANUPQData.io[<ioIndex>]' with  some  or
all of the fields listed below is  created.  Whenever  a  non-interactive
function is called (see Chapter~"Non-interactive ANUPQ  Functions"),  the
record `ANUPQData.ni' is updated with fields that, if bound, have exactly
the same purpose as for a `ANUPQData.io[<ioIndex>]' record. 

\beginitems

\quad`stream'& the  IOStream  opened  for  interactive  {\ANUPQ}  process
<ioIndex> or non-interactive {\ANUPQ} function;

\quad`group'& the group given  as  first  argument  to  `PqStart',  `Pq',
`PqEpimorphism',  `PqDescendants'  or  `PqStandardPresentation'  (or  any
synonymous methods);

\quad`haspcp'& is bound and set to `true' when a pc presentation is first
set   inside   the   `pq'   program   (e.g.~by   `PqPcPresentation'    or
`PqRestorePcPresentation'  or  a  higher  order   function   like   `Pq',
`PqEpimorphism', `PqPCover', `PqDescendants' or  `PqStandardPresentation'
that does a `PqPcPresentation' operation, but *not* `PqStart' which  only
starts up an interactive {\ANUPQ} process);

\quad`gens'& a list of the generators of the  group  `group'  as  strings
(the same as those passed to the `pq' program);

\quad`rels'& a list of the relators of the group `group' as strings  (the
same as those passed to the `pq' program);

\quad`name'& the name of the group whose pc presentation is defined by  a
call to the `pq' program (according to the `pq'  program  --  unless  you
have used the `GroupName' option (see e.g.~"Pq") or applied the  function
`SetName'  (see~"ref:SetName")  to  the  group,  the   ``generic''   name
`"[grp]"' is set as a default);

\quad`gpnum'& if not a null string, the ``number'' (i.e.~the unique label
assigned by the `pq' program) of the last descendant processed;

\quad`class'& the largest lower exponent-$p$ central class of a  quotient
group of the group (usually `group') found by a call to the `pq' program;

\quad`forder'& the factored order of the quotient group of largest  lower
exponent-$p$ central class found for the group  (usually  `group')  by  a
call to the `pq' program (this factored order is given as  a  list  `[$p$,
$n$]', indicating an order of $p^n$);

\quad`pcoverclass'&  the  lower  exponent-$p$  central   class   of   the
$p$-covering group of a $p$-quotient of the group (usually `group') found
by a call to the `pq' program;

\quad`workspace'& the workspace set for the `pq' process (either given as
a second argument to `PqStart', or set by default to 10000000);

\quad`menu'& the current menu of the `pq' process  (the  `pq'  program  is
managed by various  menus,  the  details  of  which  the  user  shouldn't
normally need to know about -- the `menu' field remembers which menu  the
`pq' process is currently ``in'');

\quad`outfname' & is the file to which `pq' output is directed, which  is
always `ANUPQData.outfile', except when option `SetupFile' is used with a
non-interactive  function,  in  which   case   `outfname'   is   set   to
`"PQ_OUTPUT"';

\quad`pQuotient' & is set to the value returned by `Pq'  (see~"Pq")  (the
field `pQepi' is also set at the same time);

\quad`pQepi'  &  is  set  to  the  value  returned   by   `PqEpimorphism'
(see~"PqEpimorphism") (the field `pQuotient' is  also  set  at  the  same
time);

\quad`pCover'  &  is  set   to   the   value   returned   by   `PqPCover'
(see~"PqPCover");

\quad`SP' & is set to the value returned by  `PqStandardPresentation'  or
`StandardPresentation'  (see~"PqStandardPresentation!interactive")   when
called interactively, for process <i> (the field `SPepi' is also  set  at
the same time);

\quad`SPepi'    &    is    set    to    the     value     returned     by
`EpimorphismPqStandardPresentation' or  `EpimorphismStandardPresentation'
(see~"EpimorphismPqStandardPresentation!interactive")     when     called
interactively, for process <i> (the field `SP' is also set  at  the  same
time);

\quad`descendants' & is    set    to    the     value     returned     by
`PqDescendants' (see~"PqDescendants");

\quad`treepos' &  if  set  by  a  call  to  `PqDescendantsTreeCoclassOne'
(see~"PqDescendantsTreeCoclassOne"), it contains  a  record  with  fields
`class', `node' and `ndes' being the information that determines the last
descendant with a non-zero number of descendants processed;

\quad`xgapsheet' & if set  by  a  call  to  `PqDescendantsTreeCoclassOne'
(see~"PqDescendantsTreeCoclassOne")  during  an   {\XGAP}   session,   it
contains the {\XGAP} `Sheet' on which the descendants tree is  displayed;
and

\quad`nextX'  &  if  set  by  a  call  to   `PqDescendantsTreeCoclassOne'
(see~"PqDescendantsTreeCoclassOne")  during  an   {\XGAP}   session,   it
contains  a  list  of  integers,  the  <i>th  entry  of  which   is   the
<x>-coordinate of the next node (representing a descendant) for the <i>th
class.

\enditems

\>ANUPQDirectoryTemporary( <dir> ) F

calls the UNIX command `mkdir' to create <dir>, which must be  a  string,
and if successful a directory  object  for  <dir>  is  both  assigned  to
`ANUPQData.tmpdir' and returned. The field  `ANUPQData.outfile'  is  also
set to be a file in `ANUPQData.tmpdir', and on exit from {\GAP} <dir>  is
removed. Most users will never need  this  command;  by  default,  {\GAP}
typically   chooses   a   ``random''   subdirectory   of    `/tmp'    for
`ANUPQData.tmpdir' which may occasionally have  limits  on  what  may  be
written there. `ANUPQDirectoryTemporary' permits the  user  to  choose  a
directory (object) where one is not so limited.

%%%%%%%%%%%%%%%%%%%%%%%%%%%%%%%%%%%%%%%%%%%%%%%%%%%%%%%%%%%%%%%%%%%%%%%%%
\Section{Setting the Verbosity of ANUPQ via Info and InfoANUPQ}

\>`InfoANUPQ' V

The input to and the output from the `pq' program  is,  by  default,  not
displayed. However the user may choose to  see  some,  or  all,  of  this
input/output.   This   is   done   via   the   `Info'   mechanism    (see
Chapter~"ref:Info Functions" in the {\GAP} Reference  Manual).  For  this
purpose, there is the <InfoClass>  `InfoANUPQ'.  If  the  `InfoLevel'  of
`InfoANUPQ' is high enough each line of `pq' input/output is directed  to
a call to `Info' and will be displayed for the user to see.  By  default,
the `InfoLevel' of `InfoANUPQ' is 1, and it is recommended that you leave
it at this level, or higher. Messages that  the  user  should  presumably
want to see and output from the `pq' program influenced by the  value  of
the option `OutputLevel' (see the options listed in Section~"Pq"),  other
than timing and memory usage are directed to `Info' at `InfoANUPQ'  level
1.

To turn off *all* `InfoANUPQ' messaging, set the `InfoANUPQ' level to 0.

There are five other user-intended `InfoANUPQ' levels: 2, 3, 4, 5 and 6.

\beginexample
gap> SetInfoLevel(InfoANUPQ, 2);
\endexample

enables the display of most timing and memory usage data  from  the  `pq'
program, and also the number of identity instances when the  `Identities'
option is used. (Some timing and memory  usage  data,  particularly  when
profuse in quantity, is `Info'-ed at `InfoANUPQ' level 3  instead.)  Note
that the the {\GAP} functions `time' and `Runtime' (see~"ref:Runtime"  in
the {\GAP} Reference Manual) count the time spent by {\GAP} and *not* the
time spent by the (external) `pq' program.

\beginexample
gap> SetInfoLevel(InfoANUPQ, 3);
\endexample

enables the display of output of the nature of the first two  `InfoANUPQ'
that was not directly invoked by the  user  (e.g.~some  commands  require
{\GAP} to discover something about the current state known  to  the  `pq'
program). The identity instances processed under the `Identities'  option
are also displayed at  this  level.  In  some  cases,  the  `pq'  program
produces a  lot  of  output  despite  the  fact  that  the  `OutputLevel'
(see~"option OutputLevel") is unset or is set to 0; such output  is  also
`Info'-ed at `InfoANUPQ' level 3.

\beginexample
gap> SetInfoLevel(InfoANUPQ, 4);
\endexample

enables the display of all the commands  directed  to  the  `pq'  program,
behind a ```ToPQ> ''' prompt (so that you can  distinguish  it  from  the
output from the `pq' program). See Section~"Hints and Warnings  regarding
the use of  Options"  for  an  example  of  how  this  can  be  a  useful
troubleshooting tool.

\beginexample
gap> SetInfoLevel(InfoANUPQ, 5);
\endexample

enables the display of the `pq' program's prompts for input. Finally,

\beginexample
gap> SetInfoLevel(InfoANUPQ, 6);
\endexample

enables the display of all other output from the `pq' program, namely  the
banner and menus. However, the timing data printed when the  `pq'  program
exits can never be observed.

%%%%%%%%%%%%%%%%%%%%%%%%%%%%%%%%%%%%%%%%%%%%%%%%%%%%%%%%%%%%%%%%%%%%%%%%%
\Section{Utility Functions}

\>PqLeftNormComm( <elts> ) F

returns for a list of elements of some group (e.g.~<elts> may be  a  list
of words in the generators of  a  free  or  fp  group)  the  left  normed
commutator of <elts>, e.g.~if <w1>, <w2>, <w3>  are  such  elements  then
`PqLeftNormComm( [<w1>, <w2>, <w3>] );' is  equivalent  to  `Comm(  Comm(
<w1>, <w2> ), <w3> );'.

*Note:* <elts> must contain at least two elements.

\>PqGAPRelators( <group>, <rels> ) F

returns a list of words that {\GAP} understands, given a list  <rels>  of
strings in the string representations of the generators of the  fp  group
<group> prepared as a list of relators for the `pq' program.

*Note:*
The `pq' program does not  use  `/'  to  indicate  multiplication  by  an
inverse and uses square brackets to represent (left normed)  commutators.
Also, even though the `pq' program accepts  relations,  all  elements  of
<rels> *must* be in relator form, i.e.~a relation of form `<w1>  =  <w2>'
must be written as `<w1>*(<w2>)^-1'.

Here is an example:

\beginexample
gap> F := FreeGroup("a", "b");
<free group on the generators [ a, b ]>
gap> PqGAPRelators(F, [ "a*b^2", "[a,b]^2*a", "([a,b,a,b,b]*a*b)^2*a" ]);
[ a*b^2, a^-1*b^-1*a*b*a^-1*b^-1*a*b*a, b^-1*a^-1*b^-1*a^-1*b*a*b^-1*a*b*a^
    -1*b*a^-1*b^-1*a*b*a*b^-1*a^-1*b^-1*a^-1*b*a*b^-1*a*b^-1*a^-1*b*a^-1*b^
    -1*a*b*a*b*a^-1*b*a*b^-1*a*b*a^-1*b*a^-1*b^-1*a*b*a*b^-1*a^-1*b^-1*a^
    -1*b*a*b^-1*a*b^-1*a^-1*b*a^-1*b^-1*a*b*a*b^2*a*b*a ]
\endexample

\>PqParseWord( <word>, <n> ) F

parses a <word>, a string  representing  a  word  in  the  pc  generators
`x1,...,x<n>',  through  {\GAP}.  This  function   is   provided   as   a
rough-and-ready check of <word> for syntax errors. A  syntax  error  will
cause the entering of a `break'-loop, in which the error message  may  or
may not be meaningful (depending on whether the syntax error gets  caught
at the {\GAP} or kernel level).

*Note:*
The reason the generators *must* be `x1,...,x<n>' is that these  are  the
pc generator names used  by  the  `pq'  program  (as  distinct  from  the
generator names for the group provided by the user  to  a  function  like
`Pq' that invokes the `pq' program).

\>PqExample() F
\>PqExample( <example>[, PqStart][, Display] ) F
\>PqExample( <example>[, PqStart][, <filename>] ) F

With no arguments,  or  with  single  argument  `"index"',  or  a  string
<example> that is not the name of a file in the `examples' directory,  an
index of available examples is displayed.

With just the one argument <example> that is the name of a  file  in  the
`examples' directory, the example contained in that file is  executed  in
its simplest form. Some examples accept options  which  you  may  use  to
modify some of the options used in the commands of the example.  To  find
out which options an example accepts,  use  one  of  the  mechanisms  for
displaying the example described below.

Some examples have both non-interactive and interactive forms; those that
are non-interactive only have a name ending  in  `-ni';  those  that  are
interactive only have a name ending in `-i'; examples with  names  ending
in  `.g'  also  have  only  one  form;  all  other  examples  have   both
non-interactive and interactive forms and for these giving  `PqStart'  as
second argument invokes `PqStart' initially  and  makes  the  appropriate
adjustments  so  that  the  example  is  executed  or   displayed   using
interactive functions.

If `PqExample' is called with last (second or third)  argument  `Display'
then the example  is  displayed  without  being  executed.  If  the  last
argument is a non-empty  string  <filename>  then  the  example  is  also
displayed without being executed but is also written to a file with  that
name. Passing an empty string as last argument has  the  same  effect  as
passing `Display'.

*Note:*
The  variables  used  in  `PqExample'  are  local  to  the   running   of
`PqExample', so there's no  danger  of  having  some  of  your  variables
over-written. However, they are not  completely  lost  either.  They  are
saved to a record `ANUPQData.examples.vars', i.e.~if `F'  is  a  variable
used in the example then you will be able to access it after  `PqExample'
has finished as `ANUPQData.examples.vars.F'.

\>AllPqExamples() F

returns  a  list  of  all  currently  available   examples   in   default
UNIX-listing (i.e.~alphabetic) order.

\>GrepPqExamples( <string> ) F

runs the UNIX command `grep <string>'  over  the  {\ANUPQ}  examples  and
returns the list of examples for which  there  is  a  match.  The  actual
matches are `Info'-ed at `InfoANUPQ' level 2.

\>ToPQLog([ <filename> ]) F

With string argument <filename>,  `ToPQLog'  opens  the  file  with  name
<filename> for logging; all commands written to the `pq' binary (that are
`Info'-ed behind a ```ToPQ> ''' prompt at `InfoANUPQ' level 4)  are  then
also written to that  file  (but  without  prompts).  With  no  argument,
`ToPQLog' stops logging to whatever file was being logged to. If  a  file
was already being logged to, that file is closed and the file  with  name
<filename> is opened for logging.

%%%%%%%%%%%%%%%%%%%%%%%%%%%%%%%%%%%%%%%%%%%%%%%%%%%%%%%%%%%%%%%%%%%%%%%%%
\Section{Attributes and a Property for fp and pc p-groups}

\>NuclearRank( <G> ) A
\>MultiplicatorRank( <G> ) A
\>IsCapable( <G> ) P

return the nuclear rank  of  <G>,  $p$-multiplicator  rank  of  <G>,  and
whether <G> is capable (i.e.~`true' if it is, or `false' if it  is  not),
respectively.

These attributes and property are set automatically if <G> is one of  the
following:

\beginlist%unordered

\item{--}  an  fp   group   returned   by   `PqStandardPresentation'   or
`StandardPresentation' (see~"PqStandardPresentation");

\item{--} the  image  (fp  group)  of  the  epimorphism  returned  by  an
`EpimorphismPqStandardPresentation' or  `EpimorphismStandardPresentation'
call (see~"EpimorphismPqStandardPresentation"); or

\item{--} one of the pc groups of the list  of  descendants  returned  by
`PqDescendants' (see~"PqDescendants").

\endlist

If <G> is an fp group or a pc $p$-group and not one of the above and  the
attribute  or  property  has  not  otherwise  been  set  for  <G>,   then
`PqStandardPresentation' is called to set  all  three  of  `NuclearRank',
`MultiplicatorRank' and `IsCapable', before returning the  value  of  the
attribute or property actually called. Such a  group  <G>  must  know  in
advance that it is a $p$-group; this is the case for the groups  returned
by the functions  `Pq'  and  `PqPCover',  and  the  image  group  of  the
epimorphism returned by `PqEpimorphism'. Otherwise, if you know the group
to be a $p$-group, then this can be set by typing

\){\kernttindent}SetFeatureObj( <G>, IsPGroup, true );

or by invoking `IsPGroup( <G> )'. Note that for  an  fp  group  <G>,  the
latter may result in a coset enumeration which might not terminate  in  a
reasonable time.

*Note:* For <G> such that `HasNuclearRank(<G>) = true',  `IsCapable(<G>)'
is equivalent to (the truth or falsity of) `NuclearRank( <G> ) = 0'.

%%%%%%%%%%%%%%%%%%%%%%%%%%%%%%%%%%%%%%%%%%%%%%%%%%%%%%%%%%%%%%%%%%%%%%%%%
\Section{Hints and Warnings regarding the use of Options}

On a first reading we recommend you skip this  section  (unless  you  are
still using {\GAP}~4.2) and come back to it if  and  when  you  run  into
trouble.

\index{menu item!of pq program}
\index{option!of pq program is a menu item}
*Note:*
By ``options'' we refer to {\GAP} options. The `pq' program also uses the
term ``option''; to distinguish the two usages  of  ``option'',  in  this
manual we use the term *menu item* to refer  to  what  the  `pq'  program
refers to as an ``option''.

Options are passed to the {\ANUPQ} interface functions in either  of  the
two usual mechanisms provided by {\GAP}, namely:

\beginlist%unordered

\item{--} options may be set globally using  the  function  `PushOptions'
(see Chapter~"ref:Options Stack" in the {\GAP} Reference Manual); or

\item{--} options may be appended to the argument list  of  any  function
call,   separated   by   a   colon   from   the   argument   list    (see
Chapter~"ref:Function Calls" in the {\GAP} Reference  Manual),  in  which
case they are then passed on recursively to any subsequent inner function
call, which may in turn have options of their own.

\endlist

Particularly,  when  one  is   using   the   interactive   functions   of
Chapter~"Interactive ANUPQ Functions",  one  should,  in  general,  avoid
using the global method of passing options. In fact,  it  is  recommended
that  prior  to  calling  `PqStart'  the  `OptionsStack'  be  empty.  The
essential problem  with  setting  options  globally  using  the  function
`PushOptions' is that options pushed onto `OptionsStack',  in  this  way,
(generally) remain there until an explicit `PopOptions()' call is made.

In contrast, options passed in the usual way behind a colon  following  a
function's arguments (see "ref:Function Calls" in  the  {\GAP}  Reference
Manual) are local, and disappear from `OptionsStack' after  the  function
has  executed  successfully.  If  the   function   does   *not*   execute
successfully, i.e.~it runs into error and the user `quit's the  resulting
`break' loop (see Section~"ref:Break  loops"  in  the  Reference  Manual)
rather than attempting to repair the problem and  typing  `return;'  then
usually  (so  long  as  your  {\GAP}  is  at  least  version   4.3)   the
`OptionsStack' is reset. If the error  was  detected  inside  the  kernel
(hopefully, this should occur only rarely, if at all) then the options of
that function will *not* be cleared from `OptionsStack'; in such cases:

\beginexample
gap> ResetOptionsStack();
#I  Options stack is already empty
\endexample

is usually necessary (see Chapter~"ref:ResetOptionsStack" in  the  {\GAP}
Reference  Manual),  which   recursively   calls   `PopOptions()'   until
`OptionsStack' is empty, or as in the  above  case  warns  you  that  the
`OptionsStack' is already empty.

Note that a function, that is passed options after the colon,  will  also
see any global options  or  any  options  passed  down  recursively  from
functions calling that function, unless those options are over-ridden  by
options passed via the function. Also, note that  duplication  of  option
names  for  different  programs  may  lead  to  misinterpretations,   and
mis-spelled options will not be ``seen''.

The   non-interactive   functions   of   Chapter~"Non-interactive   ANUPQ
Functions" that have `Pq' somewhere in their name provide an  alternative
method  of  passing  options  as  additional  arguments.  This  has   the
advantages that options can be abbreviated and mis-spelled  options  will
be trapped.

\index{troubleshooting tips}
\>`ANUPQWarnOfOtherOptions' V

is a global variable that is by default `false'. If it is set  to  `true'
then any function provided by the {\ANUPQ} function  that  recognises  at
least one option, will warn you of ``other'' options,  i.e.~options  that
the  function  does  not  recognise.  These  warnings  are   emitted   at
`InfoWarning' or `InfoANUPQ'  level  1.  This  is  useful  for  detecting
mis-spelled options. Here is an example using the  function  `Pq'  (first
described in Chapter~"Non-interactive ANUPQ functions"):

\begintt
gap> SetInfoLevel(InfoANUPQ, 1);        # Set InfoANUPQ to default level
gap> ANUPQWarnOfOtherOptions := true;;
gap> # The following makes entry into break loops very ``quiet'' ...
gap> OnBreak := function() Where(0); end;;
gap> F := FreeGroup( "a", "b" );
<free group on the generators [ a, b ]>
gap> Pq( F : Prime := 2, Classbound := 1 );
#I  ANUPQ Warning: Options: [ "Classbound" ] ignored
#I  (invalid for generic function: `Pq').
user interrupt at
moreOfline := ReadLine( iostream );
Entering break read-eval-print loop ...
you can 'quit;' to quit to outer loop, or
you can 'return;' to continue
\endtt

Here we mistyped `ClassBound'  as  `Classbound',  and  after  seeing  the
`Info'-ed warning that `Classbound' was ignored, we typed  a  <control>-C
(that's the ```user interrupt at''' message) which took us into  a  break
loop. Since the `Pq' command was not able to finish, the options  `Prime'
and `Classbound', in particular, will still be on the `OptionsStack':

\begintt
brk> OptionsStack;
[ rec( Prime := 2, Classbound := 1 ), 
  rec( Prime := 2, Classbound := 1, PqEpiOrPCover := "pQuotient" ) ]
\endtt

The option `PqEpiOrPCover' is a behind-the-scenes option  that  need  not
concern the user. Since {\GAP} 4.3, on `quit'ting  the  `break'-loop  the
`OptionsStack' is reset and a warning telling you this is emitted:

\begintt
brk> quit; # to get back to the `gap>' prompt
#I  Options stack has been reset
\endtt

However, as mentioned above, if your version of {\GAP} is only  4.2  then
you will need to explicitly remove these options, by:

\begintt
gap> ResetOptionsStack();
\endtt

Note that there have been many improvements to {\GAP} since  {\GAP}  4.2,
particularly  in  its  iostream  functionality.  You  will   get   strong
performance benefits by upgrading to the most recent version of {\GAP}.

Above, we altered `OnBreak' (see~"ref:OnBreak" in the  Reference  manual)
to reduce the back-tracing on entry into a break  loop.  We  now  restore
`OnBreak' to its usual value.

\beginexample
gap> OnBreak := Where;;
\endexample

*Notes*

In cases where functions recursively call others with options  (e.g.~when
using `PqExample'  with  options),  setting  `ANUPQWarnOfOtherOptions  :=
true' may give rise to spurious ``other'' option detections.

It is recommended that the novice user set  `ANUPQWarnOfOtherOptions'  to
`true' in their `.gaprc' file (see Section~"Loading the ANUPQ Package").

*Other Troubleshooting Strategies*

There are some other strategies which may have helped us to see our error
above. The function `Pq' recognises the option `OutputLevel' (see~"option
OutputLevel"); if this option is set to at  least  1,  the  `pq'  program
provides information on each class quotient as it is generated:

\begintt
gap> ANUPQWarnOfOtherOptions := false;; # Set back to normal
gap> F := FreeGroup( "a", "b" );;
gap> Pq( F : Prime := 2, Classbound := 1, OutputLevel := 1 ); 
#I  Lower exponent-2 central series for [grp]
#I  Group: [grp] to lower exponent-2 central class 1 has order 2^2
#I  Group: [grp] to lower exponent-2 central class 2 has order 2^5
#I  Group: [grp] to lower exponent-2 central class 3 has order 2^10
#I  Group: [grp] to lower exponent-2 central class 4 has order 2^18
#I  Group: [grp] to lower exponent-2 central class 5 has order 2^32
#I  Group: [grp] to lower exponent-2 central class 6 has order 2^55
#I  Group: [grp] to lower exponent-2 central class 7 has order 2^96
#I  Group: [grp] to lower exponent-2 central class 8 has order 2^167
#I  Group: [grp] to lower exponent-2 central class 9 has order 2^294
#I  Group: [grp] to lower exponent-2 central class 10 has order 2^520
#I  Group: [grp] to lower exponent-2 central class 11 has order 2^932
#I  Group: [grp] to lower exponent-2 central class 12 has order 2^1679
[... output truncated ...]
\endtt

After seeing the information for the class 2 quotient we may have got the
idea that the  `Classbound'  option  was  not  recognised  and  may  have
realised that this was due to a mis-spelling. The above  will  ordinarily
cause   the   available   space   to    be    exhausted,    necessitating
user-intervention by typing <control>-C and `quit;' (to escape the  break
loop); otherwise `Pq' terminates when the class reaches 63  (the  default
value of `ClassBound').

If you have some familiarity with ``keyword'' command input to  the  `pq'
binary, then setting the level  of  `InfoANUPQ'  to  4  would  also  have
indicated a problem:

\begintt
gap> ResetOptionsStack(); # Necessary, if a break-loop was entered above
gap> SetInfoLevel(InfoANUPQ, 4);
gap> Pq( F : Prime := 2, Classbound := 1 );                  
#I  ToPQ> 7  #to (Main) p-Quotient Menu
#I  ToPQ> 1  #define group
#I  ToPQ> name [grp]
#I  ToPQ> prime 2
#I  ToPQ> class 63
#I  ToPQ> exponent 0
#I  ToPQ> output 0
#I  ToPQ> generators { a,b }
#I  ToPQ> relators   {  };
[... output truncated ...]
\endtt

Here the line ```\#I  ToPQ> class 63''' indicates that a directive to set
the classbound to 63 was sent to the `pq' program.

%%%%%%%%%%%%%%%%%%%%%%%%%%%%%%%%%%%%%%%%%%%%%%%%%%%%%%%%%%%%%%%%%%%%%%%%%
%%
%E
