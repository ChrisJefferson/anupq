%%%%%%%%%%%%%%%%%%%%%%%%%%%%%%%%%%%%%%%%%%%%%%%%%%%%%%%%%%%%%%%%%%%%%%%%%
%%
%W  intro.tex       ANUPQ documentation - introduction      Werner Nickel
%%
%%
%H  $Id$
%%
%%

%%%%%%%%%%%%%%%%%%%%%%%%%%%%%%%%%%%%%%%%%%%%%%%%%%%%%%%%%%%%%%%%%%%%%%%%%
\Chapter{Introduction}

The {\GAP} 4 share package {\ANUPQ} provides an interface to the ANU PQ C
progam `pq' written by Eamonn O'Brien, making the functionality of the  C
program available to {\GAP}. Henceforth, we shall refer to  the  {\ANUPQ}
interface (resp. the `pq' binary), when discussing the  {\GAP}  component
(resp. the C code component) of the  {\ANUPQ}  share  package.  The  `pq'
program consists of implementations of the following algorithms:

\beginlist

\item{1.}
A *$p$-quotient algorithm* to compute power-commutator presentations  for
$p$-factor groups of finitely presented groups.

The algorithm implemented here is based on that described by  Newman  and
O'Brien \cite{NO96}, Havas and Newman~\cite{HN80}, and papers referred to
there. Another description of  the  algorithm  is  given  by  Vaughan-Lee
(see~\cite{Vau90a}, \cite{Vau90b}).  A  FORTRAN  implementation  of  this
algorithm was programmed by Alford and Havas. The basic  data  structures
of that implementation are retained.

\item{2.} 
A  *$p$-group  generation   algorithm*   to   generate   power-commutator
presentations of groups of prime power order.

The algorithm implemented here is based on the  algorithms  described  by
Newman~\cite{New77} and O'Brien~\cite{OBr90}. A FORTRAN implementation of
this algorithm was developed earlier by Newman and O'Brien.

\item{3.}
A  *standard  presentation  algorithm*  used  to  compute   a   canonical
power-commutator presentation of a $p$-group.

The algorithm implemented here is described in~\cite{OBr94}.

\item{4.} 
An algorithm which can be used to compute the *automorphism group*  of  a
$p$-group.

The algorithm implemented here is described in~\cite{OBr94}. 
This part of  the  standalone  program  is  not  accessible  through  the
{\ANUPQ} share package.  Instead,  users  are  advised  to  consider  the
{\GAP}~4 share package {\AutPGrp}, which implements a better algorithm in
{\GAP} for the computation of automorphism groups of $p$-groups.

\endlist

Further   background   may   be   found   in~\cite{OBr95},   \cite{Vau84}
and~\cite{NNN98}.

For details regarding the standalone version see the file `guide.dvi'.

%%%%%%%%%%%%%%%%%%%%%%%%%%%%%%%%%%%%%%%%%%%%%%%%%%%%%%%%%%%%%%%%%%%%%%%%%
\Section{How to read this manual}

\index{ANUPQ}
Instructions on how to install and load the {\ANUPQ}  share  package  are
located in Sections~"Installing the ANUPQ Share Package" and~"Loading the
ANUPQ Share Package", respectively.

*need to fill in some details here about the structure of the manual
 or perhaps this section should come first*

%%%%%%%%%%%%%%%%%%%%%%%%%%%%%%%%%%%%%%%%%%%%%%%%%%%%%%%%%%%%%%%%%%%%%%%%%
\Section{Authors}

The C implementation of the ANU `pq' standalone was developed by

\begintt
Eamonn O'Brien
Department of Mathematics
University of Auckland
Private Bag 92019
Auckland
New Zealand
\endtt

{\kernttindent}`email:' \Mailto{obrien@math.auckland.ac.nz}

The {\GAP} 4 version of this package was adapted from the {\GAP} 3
version by  

\begintt
Werner Nickel
AG 2, Fachbereich Mathematik, TU Darmstadt
Schlossgartenstr. 7, 64289 Darmstadt, Germany
\endtt

{\kernttindent}`email:' \Mailto{nickel@mathematik.tu-darmstadt.de}

%%%%%%%%%%%%%%%%%%%%%%%%%%%%%%%%%%%%%%%%%%%%%%%%%%%%%%%%%%%%%%%%%%%%%%%%%
%%
%E
