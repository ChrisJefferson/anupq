%%%%%%%%%%%%%%%%%%%%%%%%%%%%%%%%%%%%%%%%%%%%%%%%%%%%%%%%%%%%%%%%%%%%%%%%%
%%
%W  basics.tex      ANUPQ documentation - background info   Werner Nickel
%W                                                      Joachim Neubueser
%%
%H  $Id$
%%
%%

%%%%%%%%%%%%%%%%%%%%%%%%%%%%%%%%%%%%%%%%%%%%%%%%%%%%%%%%%%%%%%%%%%%%%%%%%
\Chapter{Mathematical Background and Terminology}

In this chapter  we will give a brief  description of the mathematical
notions used in the algorithms  implemented in the ANU `pq' program
that are made accessible from {\GAP} through this package.  For proofs
and  details  we  will  point  to relevant  places  in  the  published
literature.  Also we will try  to give some explanation of terminology
that may help to use the ``low-level'' interactive functions described
in Section~"Low-level Interactive ANUPQ  Functions based on menu items
of the pq program".

%%%%%%%%%%%%%%%%%%%%%%%%%%%%%%%%%%%%%%%%%%%%%%%%%%%%%%%%%%%%%%%%%%%%%%
\Section{Basic notions}

*pc presentations and consistency*

For details, see \cite{NNN98}.

Every $p$-group $G$ has a presentation of the form: 
$$ 
\{a_1,\dots, a_n \mid a_i^p = v_{ii}, 1 \le i \le n, 
                   [a_k, a_j] = v_{jk}, 1 \le j \< k \le n \}.  
$$
where $v_{jk}$is a word in the elements $a_{k+1},\dots,a_n$ for 
$1 \le j \< k \le n$.

\index{power-commutator presentation}\index{pc presentation}\index{pcp}
\index{pc generators}\index{collection}
This is  called a *power-commutator*  presentation (or *pc presentation*
or *pcp*) of $G$, generators from  such a presentation will be referred
to as *pc generators*.  In terms of such pc generators every element
of $G$  can be written in a  ``normal form'' $a_1^{e_1},\dots,a_n^{e_n}$
with $0 \le  e_i \< p$.  Moreover any given  product of the generators
and  their inverses can  be brought  into such a  normal form  using the
defining  relations in the  above presentation  as rewrite  rules. Any
such process is called  *collection*.  For the discussion of various
collection methods see \cite{LGS90}.

\index{consistent}\index{confluent rewriting system}\index{confluent}
Every  $p$-group of order  $p^n$ has  such a  pcp presentation  on $n$
generators and conversely every such presentation defines a $p$-group.
However a $p$-group defined by a pc-presentation on $n$ generators can
be of  smaller order $p^m$ with  $m\<n$. A pcp on  $n$ generators that
does  in   fact  define   a  $p$-group  of   order  $p^n$   is  called
*consistent* in this  manual, in line with most  of the literature on
the  algorithms  occurring  here.  A  consistent  pcp   determines   a
*confluent rewriting system* (see~"ref:IsConfluent") for the group  it
defines and for  this  reason  often  (in  particular  in  the  {\GAP}
Reference  Manual)  such   a   pcp   presentation   is   also   called
*confluent*.

Consistency of a pcp is tantamount to the fact that for any given word
in the generators any two collections will yield the same normal form.

\index{consistency conditions}
Consistency of a  pcp can be checked by a  finite set of *consistency
conditions*, demanding  that collection of the left hand  side and of
the right  hand side of  certain equations, starting  with subproducts
indicated by bracketing, will result in the same normal form.  There
are 3 types of such equations (that will be referred to in the manual):
$$
\matrix{
(a^n)a &=& a(a^n)                           \hfill&{\rm (Type\ 1)}\cr
(b^n)a &=& b^(n-1)(ba), b(a^n) = (ba)a^(n-1)\hfill&{\rm (Type\ 2)}\cr
  (ba) &=& (cb)a                            \hfill&{\rm (Type\ 3)}\cr
}
$$

*Exponent-$p$ central series and weighted pc presentations*

For details, see \cite{NNN98}.

\atindex{exponent-p central series}{@exponent-$p$ central series}
The (*descending*  or  *lower*)  (exponent-)$p$-central  series 
of an arbitrary  group $G$ is defined by  
$$ 
U_1  := G,  U_i := [G, U_{i-1}] U_{i-1}^p.   
$$ 
\atindex{p-class}{@$p$-class}\index{class}
For a $p$-group $G$ this  series terminates with the trivial group. $G$
has $p$-class $c$  if $c$ is the smallest  integer such that $U_{c+1}$
is  the trivial group.   In this  manual, as  well as  in much  of the
literature about the pq- and related algorithms the $p$-class is often
referred to simply as *class*.

Let  the  $p$-group $G$  have  a consistent  pcp  as  above. Then  the
subgroups
$$
\langle1\rangle \< \langle a_1 \rangle \< \langle a_1, a_2 \rangle %
    \< \dots \< \langle a_1,\dots,a_i \rangle \< \dots \< G
$$
\index{weight function}
form a central series  of $G$. If this refines  the $p$-central series,
we can  define the  *weight function* $w$  for the pc  generators by
$w(a_i) = k$ if $a_i$ is contained in $U_{k-1}$  but not in $U_k$.

\index{weighted pcp}
The pair of  such a weight function and  a pcp allowing it  is called a
*weighted pcp*.

*$p$-cover, $p$-multiplicator*

For details, see \cite{NNN98}.

\atindex{p-covering group}{@$p$-covering group}\atindex{p-cover}{@$p$-cover}
\atindex{p-multiplicator}{@$p$-multiplicator}
\atindex{p-multiplicator rank}{@$p$-multiplicator rank}
\index{multiplicator rank}
Let $d$ be the minimal number of generators of the $p$-group G. Then G
is isomorphic to a factor group $F/R$ of a free group $F$ of rank $d$.
It can be proved (see e.g.~\cite{OBr90}) that the isomorphism type  of
$F/[F, R]  R^p$  depends  only  on  $G$.  This  group  is  called  the
*$p$-covering group* or *$p$-cover* of $G$, and  $R/[F,  R]  R^p$  the
$p$-multiplicator of  $G$.  The  $p$-multiplicator  is  of  course  an
elementary abelian $p$-group, its  minimal  number  of  generators  is
called the *($p$-)multiplicator rank*.

%%%%%%%%%%%%%%%%%%%%%%%%%%%%%%%%%%%%%%%%%%%%%%%%%%%%%%%%%%%%%%%%%%%%%
\Section{The p-quotient algorithm}

The aim of the $p$-quotient (`pq'-) algorithm is  to  find  $p$-factor
groups (or $p$-quotient groups) of a  given  finitely  presented  (fp)
group $G$ for a given prime $p$.

The `pq'  algorithm successively determines  the factor groups  of the
groups of the $p$-central series of a fp group $G$. If a bound $b$ for
the $p$-class is given, it will determine those factor groups up to at
most  $p$-class  $b$, If  the  $p$-central  series  terminates with  a
subgroup $U_k$ with $k \< b$, the algorithm will stop with that group.
If  no such  bound is  given, it  will try  to find  the  biggest such
factorgroup.

$G/U_2$ is the largest elementary  abelian $p$-factor group of $G$ and
this  can  be found  from  the relation  matrix  of  $G$ using  matrix
diagonalisation modulo $p$. So  it suffices to explain how $G/U_{i+1}$
is found from $G/U_i$ for some $i \ge 1$.

*UP TO HERE*

If they are  not fulfilled they will provide a  set of equations among
the 

%%%%%%%%%%%%%%%%%%%%%%%%%%%%%%%%%%%%%%%%%%%%%%%%%%%%%%%%%%%%%%%%%%%%%
\Section{Old stuff}

\beginlist

\item{1.}
A $p$-quotient  algorithm  to  compute  pc-presentations  for  $p$-factor
groups of finitely presented groups. The algorithm  implemented  here  is
based on that described by Newman  and  O'Brien  \cite{NO96},  Havas  and
Newman~\cite{HN80}, and papers referred to there. Another description  of
the algorithm is given by Vaughan-Lee (see~\cite{Vau90a}, \cite{Vau90b}).
A FORTRAN implementation of this algorithm was programmed by  Alford  and
Havas. The basic data structures of that implementation are retained.

\item{2.} 
A $p$-group generation algorithm to generate pc-presentations  of  groups
of prime power order. The algorithm implemented  here  is  based  on  the
algorithms described by Newman~\cite{New77} and  O'Brien~\cite{OBr90}.  A
FORTRAN implementation of this algorithm was developed earlier by  Newman
and O'Brien.

\item{3.}
A  standard  presentation  algorithm  used   to   compute   a   canonical
pc-presentation  of  a  $p$-group.  The  algorithm  implemented  here  is
described in~\cite{OBr94}.

\item{4.} 
An algorithm which can be used to compute the  automorphism  group  of  a
$p$-group. The algorithm implemented here is  described  in~\cite{OBr94}.
This part of  the  standalone  program  is  not  accessible  through  the
{\ANUPQ} package. Instead, users are advised  to  consider  the  {\GAP}~4
package {\AutPGrp}, which implements a better algorithm in {\GAP} for the
computation of automorphism groups of $p$-groups.

\endlist

Further   background   may   be   found   in~\cite{OBr95},   \cite{Vau84},
\cite{NNN98}, and~\cite{Sims94}.

For details regarding the standalone version see the file `guide.dvi'.

\index{capable}
We say a $p$-quotient descendant is *capable* if that  descendant  occurs
as the  factor  group  of  a  $p$-group  modulo  the  last  term  of  its
$p$-central series.

%%%%%%%%%%%%%%%%%%%%%%%%%%%%%%%%%%%%%%%%%%%%%%%%%%%%%%%%%%%%%%%%%%%%%%%%%
%%
%E
