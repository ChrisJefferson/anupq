%%%%%%%%%%%%%%%%%%%%%%%%%%%%%%%%%%%%%%%%%%%%%%%%%%%%%%%%%%%%%%%%%%%%%%%%%
%%
%W  install.tex     ANUPQ documentation - installation      Werner Nickel
%%
%%
%H  $Id$
%%
%%

%%%%%%%%%%%%%%%%%%%%%%%%%%%%%%%%%%%%%%%%%%%%%%%%%%%%%%%%%%%%%%%%%%%%%%%%%%%%%
\Chapter{Installing the ANUPQ Package}

The ANU  `pq' program is  written in C  and the package can  be installed
under UNIX  and in environments  similar to UNIX.  It has been  tested on
DECstation running Ultrix, a HP 9000/700 and HP 9000/800 running HP-UX, a
MIPS running RISC/os Berkeley, a  NeXTstation running NeXTSTEP 3.0, a SUN
running  SunOS and an  Intel Pentium  based PC  running Linux  or Windows
equipped with cygwin.

To install the {\ANUPQ} package, move the file `anupq-<XXX>.zoo' for some
version number <XXX> into the  `pkg'  directory  in  which  you  plan  to
install {\ANUPQ}. Usually, this  will  be  the  directory  `pkg'  in  the
hierarchy of your version of {\GAP}~4; it is  however  also  possible  to
keep an additional `pkg' directory in your private directories. The  only
essential difference  with  installing  {\ANUPQ}  in  a  `pkg'  directory
different to the {\GAP}~4 home directory is that one  must  start  {\GAP}
with the `-l' switch (see Section~"ref:Command  Line  Options"),  e.g.~if
your private `pkg' directory is a subdirectory of `mygap'  in  your  home
directory you might type:

%begintt
\){\kernttindent}gap -l ";<myhomedir>/mygap"
%endtt

where <myhomedir> is the  path  to  your  home  directory,  which  (since
{\GAP}~4.3) may be replaced  by  a  tilde.  The  empty  path  before  the
semicolon is  filled  in  by  the  default  path  of  the  {\GAP}~4  home
directory.

Then, in your chosen `pkg' directory, unzoo `anupq-<XXX>.zoo' by

%\begintt
\){\kernttindent}unzoo -x anupq-<XXX>
%\endtt

Change directory to the newly created `anupq' directory. Now you need  to
call `configure <path>' where <path> is  the  path  to  the  {\GAP}  home
directory. So for example if you install the package in  the  main  `pkg'
directory call

\begintt
./configure ../..
\endtt

What this does is look for a file `sysinfo.gap' in the root directory  of
{\GAP} in order to determine an architecture name for the subdirectory of
`bin' in which to put the compiled `pq' binary. This only makes sense  if
{\GAP} was compiled for the same architecture that `pq' will be.  If  you
have a shared file system mounted across  different  architectures,  then
you should run `configure' and `make' for {\ANUPQ} for each  architecture
immediately after compiling {\GAP} on the same architecture.

If you had to install the package in your own directory but wish  to  use
the system {\GAP}~4 (make sure it's at  least  {\GAP}~4.3fix4)  then  you
will need to find out what <path> is. To do this,  start  up  {\GAP}  and
find out what {\GAP}'s root  path  is  from  finding  the  value  of  the
variable `GAPInfo.RootPaths'  (in  {\GAP}~4.4)  or  `GAP_ROOT_PATHS'  (in
{\GAP}~4.3), e.g.

\begintt
gap> GAPInfo.RootPaths;
[ "/usr/local/lib/gap4r4/" ]
\endtt

would tell you to use `/usr/local/lib/gap4r4' for <path>.

The `configure' command will fetch the architecture type for which {\GAP}
has been compiled  last,  create  a  `Makefile'  and  list  a  number  of
``targets'' to call `make' with. If you have one of  the  standard  Linux
(or NetBSD or FreeBSD) systems with `gcc', wish  to  compile  with  `-O2'
optimisation, and have `gmp'  with  its  include  and  library  files  in
`/usr/local/include' and  `/usr/local/lib',  respectively,  you  can  now
simply call

\begintt
make
\endtt

to compile the binary and to install it in the appropriate place.

If you need a special target (perhaps you don't have `gmp' or you are not
on a Linux, NetBSD or FreeBSD system) then you need to call `make' with a
target argument. If  the  targets  displayed  on  the  screen  after  the
`configure' step rushed past your eyes and you can't scroll back  to  see
them, you can ``pipe'' those same targets through `less' or `more',  e.g.
with `more':

\begintt
make unknown || more
\endtt

An abbreviation of the target list is as follows:

\begintt
'linux-iX86-gcc2-gmp'      for IBM x86 PCs under linux/BSD with GNU cc 2 and mp
'linux-iX86-cc-gmp'        for IBM x86 PCs under linux/BSD with cc and GNU mp
'linux-iX86-gcc2'          for IBM x86 PCs under linux/BSD with GNU cc 2
'linux-iX86-cc'            for IBM x86 PCs under linux/BSD with cc (GNU)
[... 16 lines deleted ...]
'sunos-gcc2-gmp'           for SunOS with GNU cc 2 and gmp
'sunos-cc-gmp'             for SunOS with cc and GNU mp
'sunos-gcc2'               for SunOS with GNU cc 2
'sunos-cc'                 for SunOS with cc
'unix-gmp'                 for a generic unix system with cc and GNU mp
'unix'                     for a generic unix system with cc
'clean'                    remove all created files

   targets are listed according to preference,
   i.e., 'sunos-gcc2' is better than 'sunos-cc'
   no target is the same as choosing 'linux-iX86-gcc2-gmp'
   additional C compiler and linker flags can be passed with
   'make <target> COPTS=<compiler-opts> LOPTS=<linker-opts>',
   e.g., 'make sunos-cc COPTS=-g LOPTS=-g'.

   set GAP if GAP4 is not started with the command 'gap',
   e.g., 'make sunos-cc GAP=/usr/local/bin/gap4'.

   in order to use the GNU multiple precision (gmp) set
   'GNUINC' (default '/usr/local/include') and 
   'GNULIB' (default '/usr/local/lib')

   do 'make unknown || more' to see these targets again via more
\endtt

Let's suppose that the `linux-iX86-gcc2-gmp' target does not satisfy your
requirements; let's suppose your system is Solaris 2.8 (i.e.~SunOS  5.8),
you have `gmp' but its include  and  library  directories  are  somewhere
else, and that `gap4' is the command used to initiate {\GAP}~4. Then  the
following `make' call might be correct in this case:

\begintt
make sunos-gcc2-gmp GAP=gap4 GNUINC=/opt/local/include GNULIB=/opt/local/lib
\endtt

If you  don't  have  the  *GNU*  multiple  precision  arithmetic  (`gmp')
installed on your system, don't worry,  `gmp'  is  *not  required*;  just
select an appropriate target without `-gmp'.

\indextt{ANUPQ_GAP_EXEC!environment variable}
The path of {\GAP} (see *Note* below) used by the `pq' binary (the  value
`GAP' is set to in the `make' command) may be over-ridden by setting  the
environment variable `ANUPQ_GAP_EXEC'. These values are only of  interest
when the `pq' program is run  as  a  standalone;  however,  the  `testPq'
script assumes you have set one of these correctly (see  Section~"Testing
your ANUPQ installation"). When the `pq' program is started  from  {\GAP}
communication occurs via an iostream, so that the `pq'  binary  does  not
actually need to know a valid path for {\GAP} is this case.

*Note.* By ``path of {\GAP}'' we mean the path of  the  command  used  to
invoke {\GAP} (which  should  be  a  script,  e.g.  the  `gap.sh'  script
generated in the `bin' directory for the version of  {\GAP}  when  {\GAP}
was compiled). The usual strategy is to copy the  `gap.sh'  script  to  a
standard location, e.g. `/usr/local/bin/gap'. It is a mistake to copy the
{\GAP}  executable  `gap'   (in   a   directory   with   name   of   form
`bin/<compile-platform>')  to  the  standard   location,   since   direct
invocation of the executable results in  {\GAP}  starting  without  being
able to find its own library (a fatal error).

%%%%%%%%%%%%%%%%%%%%%%%%%%%%%%%%%%%%%%%%%%%%%%%%%%%%%%%%%%%%%%%%%%%%%%%%%
\Section{Testing your ANUPQ installation}

\indextt{ANUPQ_GAP_EXEC!environment variable}
Now it is time to test the  installation.  After  doing  `configure'  and
`make' you will have a `testPq' script. The script assumes that,  if  the
environment variable `ANUPQ_GAP_EXEC' is set, it is a  correct  path  for
{\GAP}, or otherwise that the `make' call that compiled the  `pq' program
set `GAP' to a correct path  for  {\GAP}  (see  Section~"Running  the  pq
program as a standalone" for more details). To run the tests, just type:

\begintt
testPq
\endtt

Some of the tests the script runs take a while. Please be  patient.  Note
that since version 1.2, the {\ANUPQ} package  requires  at  least  {\GAP}
4.3fix4 and (at  least)  version  1.1  of  the  {\AutPGrp}  package  (the
`testPq' script checks  for  these  conditions  and  hopefully  gives  an
informative message if they are not met). The output you  see  should  be
something like the following:

\begintt
Made dir: /tmp/testPq
Testing installation of ANUPQ Package (version 2.0)
 
The first two tests check that the pq C program compiled ok.
Testing the pq binary ... OK.
Testing the pq binary's stack size ... OK.
The pq C program compiled ok!
 
The next tests check that you have the right version of GAP
for version 2.0 of the ANUPQ package and that GAP is finding
the right versions of the ANUPQ and AutPGrp packages.
 
Checking GAP ...
 pq binary made with GAP set to: /usr/local/bin/gap
 Starting GAP to determine version and package availability ...
  GAP version (4.4) ... OK.
  GAP found ANUPQ package (version 2.0) ... good.
  GAP found AutPGrp package (version 1.2) ... good.
 GAP is OK.
 
Checking the link between the pq binary and GAP ... OK.
Testing the standard presentation part of the pq binary ... OK.
Doing p-group generation (final GAP/ANUPQ) test ... OK.
Tests complete.
Removed dir: /tmp/testPq
Enjoy using your functional ANUPQ package!
\endtt

%%%%%%%%%%%%%%%%%%%%%%%%%%%%%%%%%%%%%%%%%%%%%%%%%%%%%%%%%%%%%%%%%%%%%%%%%
\Section{Running the pq program as a standalone}

\indextt{ANUPQ_GAP_EXEC!environment variable}
When the `pq' program is run as a standalone it sometimes  needs  to call
{\GAP} to compute stabilisers of subgroups; in doing so, it first  checks
the value of the environment variable `ANUPQ_GAP_EXEC', and uses that, if
set, or otherwise the value of `GAP' it was compiled with,  as  the  path
for  {\GAP}.  If  you  ran  `testPq'  (see  Section~"Testing  your  ANUPQ
installation") and you got both {\GAP} is `OK' and the link  between  the
`pq' binary and {\GAP} is `OK', you should be fine.  Otherwise  heed  the
recommendations of the error messages you get and run the `testPq'  until
all tests are passed.

It is especially important that the {\GAP}, whose path you  gave,  should
know where to find the {\ANUPQ} and {\AutPGrp} packages. To  ensure  this
the path should be to a shell script that invokes {\GAP}. If  you  needed
to install the needed packages in your own directory (because,  say,  you
are not a system administrator) then you should  create  your  own  shell
script that runs {\GAP} with a correct setting of the `-l' option and set
the path used by the `pq' binary to the path of that  script.  To  create
the script that runs {\GAP} it is easiest to copy the system one and edit
it, e.g. start by executing the following UNIX commands (skip the  second
step if you already have a `bin'  directory;  `you@unix>'  is  your  UNIX
prompt):

\begintt
you@unix> cd
you@unix> mkdir bin
you@unix> cd bin
you@unix> which gap
/usr/local/bin/gap
you@unix> cp /usr/local/bin/gap mygap
you@unix> chmod +x mygap
\endtt

At the second-last step use the path of {\GAP} returned by `which gap'.
Now hopefully you will have a copy of the script  that  runs  the  system
{\GAP} in `mygap'. Now use your favourite editor to edit the `-l' part of
the last line of `mygap' which should initially look something like:

\begintt
exec $GAP_DIR/bin/$GAP_PRG -m $GAP_MEM -o 970m -l $GAP_DIR $*
\endtt

so that it becomes (the tilde  is  a  UNIX  abbreviation  for  your  home
directory):

\begintt
exec $GAP_DIR/bin/$GAP_PRG -m $GAP_MEM -o 970m -l "$GAP_DIR;~/gapstuff" $*
\endtt

assuming that your personal {\GAP} `pkg' directory is a  subdirectory  of
`gapstuff' in your home directory. Finally, to let the `pq' program  know
where {\GAP} is and also know where your `pkg' directory is that contains
{\ANUPQ}, set the environment variable `ANUPQ_GAP_EXEC' to  the  complete
(i.e. absolute) path of  your  `mygap'  script  (do  not  use  the  tilde
abbreviation), or at the `make' step that compiles `pq' do

%begintt
\){\kernttindent}make GAP=<absolute-path-of-mygap>
%endtt

%%%%%%%%%%%%%%%%%%%%%%%%%%%%%%%%%%%%%%%%%%%%%%%%%%%%%%%%%%%%%%%%%%%%%%%%%
%%
%E
