%%%%%%%%%%%%%%%%%%%%%%%%%%%%%%%%%%%%%%%%%%%%%%%%%%%%%%%%%%%%%%%%%%%%%%%%%
%%
%W  intro.tex       ANUPQ documentation - introduction      Werner Nickel
%%
%%
%H  $Id$
%%
%%

%%%%%%%%%%%%%%%%%%%%%%%%%%%%%%%%%%%%%%%%%%%%%%%%%%%%%%%%%%%%%%%%%%%%%%%%%
\Chapter{Introduction}

\index{ANUPQ}
The {\GAP}~4 package {\ANUPQ} provides an interface to  the  ANU  `pq'  C
progam written by Eamonn O'Brien,  making  the  functionality  of  the  C
program available to {\GAP}. Henceforth, we shall refer to  the  {\ANUPQ}
package when referring to the {\GAP}  interface,  and  to  the  ANU  `pq'
program or just `pq' when referring to that C program.

The `pq' program consists of implementations of the following algorithms:

\beginlist%ordered

\item{1.}
A *$p$-quotient algorithm* to  compute  pc-presentations  for  $p$-factor
groups of finitely presented groups.

%The algorithm implemented here is based on that described by  Newman  and
%O'Brien \cite{NO96}, Havas and Newman~\cite{HN80}, and papers referred to
%there. Another description of  the  algorithm  is  given  by  Vaughan-Lee
%(see~\cite{VL90a}, \cite{VL90b}).  A  FORTRAN  implementation  of  this
%algorithm was programmed by Alford and Havas. The basic  data  structures
%of that implementation are retained.

\item{2.} 
A *$p$-group generation algorithm* to generate pc presentations of groups
of prime power order.

%The algorithm implemented here is based on the  algorithms  described  by
%Newman~\cite{New77} and O'Brien~\cite{OBr90}. A FORTRAN implementation of
%this algorithm was developed earlier by Newman and O'Brien.

\item{3.}
A  *standard  presentation  algorithm*  used  to  compute   a   canonical
pc-presentation of a $p$-group.

%The algorithm implemented here is described in~\cite{OBr94}.

\item{4.} 
An algorithm which can be used to compute the *automorphism group*  of  a
$p$-group.

\item{}
This part of the `pq' program is  not  accessible  through  the  {\ANUPQ}
package. Instead, users are advised  to  consider  the  {\GAP}~4  package
{\AutPGrp} by Bettina Eick and Eamonn O'Brien, which implements a  better
algorithm in  {\GAP}  for  the  computation  of  automorphism  groups  of
$p$-groups.
%The algorithm implemented here is described in~\cite{OBr94}.

\endlist

%Further   background   may   be   found   in~\cite{OBr95},   \cite{VL84}
%and~\cite{NNN98}.

The manual of the {\ANUPQ} package has been written for {\GAP}  4.3.  The
current  version  requires  {\GAP}  4.3  updated  to  (at  least)   fix4.
References to {\GAP} 4.2 remain, since the previous  version  was  {\GAP}
4.2 compatible. (See "Change history" for more details.)
%Nevertheless, the {\ANUPQ} package is compatible  with  {\GAP}  4.2,  but
%since it uses the  iostream  technology  introduced  in  {\GAP}  4.2,  it
%requires at least {\GAP} 4.2.

*How to read this manual*

It is not expected that readers of this manual will read it in  a  linear
fashion from cover to cover; some sections contain material that  is  far
too technical to be absorbed on a first reading.

Firstly,  installers  of  the  {\ANUPQ}  package  will   need   to   read
Chapter~"Installing the ANUPQ package", if they have not already  gleaned
these details from the `README' file.

Once the {\ANUPQ} package is installed, users  of  the  {\ANUPQ}  package
will benefit most by first reading Chapter~"Mathematical  Background  and
Terminology", which gives a  brief  description  of  the  background  and
terminology used (this chapter also cites  a  number  of  references  for
further reading), and the introduction of Chapter~"Infrastructure"  (skip
the remainder of the chapter on a first reading).

Then  the  user/reader  should  pursue   Chapter~"Non-interactive   ANUPQ
functions" in detail, delving into Chapter~"ANUPQ Options"  as  necessary
for the options of the functions that are described. The user will become
best acquainted with the {\ANUPQ} package by trying  the  examples.  This
chapter describes the non-interactive functions of the {\ANUPQ}  package,
i.e.~``one-shot'' functions that invoke the `pq' program in  such  a  way
that once {\GAP} has got what it needs, the `pq' is allowed to  exit.  It
is expected that most of the time, users will only need these functions.

Advanced users will want to explore Chapter~"Interactive ANUPQ functions"
which describes all the interactive functions of  the  {\ANUPQ}  package;
these are functions that  extract  information  via  a  dialogue  with  a
running `pq' process. Occasionally, a user needs the ``next  step'';  the
functions provided in this chapter make use of data from  previous  steps
retained by the `pq' program, thus allowing the user to interact with the
`pq' program like one can when one uses the `pq' program as a stand-alone
(see~`guide.dvi' in the `standalone-doc' directory).

After   having   read    Chapters~"Non-interactive    ANUPQ    functions"
and~"Interactive ANUPQ functions", cross-references will have  taken  the
reader into Chapter~"ANUPQ Options"; by this stage, the reader need  only
read the introduction of Chapter~"ANUPQ Options".

After the reader has developed some facility with the  {\ANUPQ}  package,
she should explore the examples described in Appendix~"Examples".

If  you  run  into   trouble   using   the   {\ANUPQ}   functions,   some
troubleshooting hints are given in Section~"Hints and Warnings  regarding
the use of Options".  If  you  are  still  using  {\GAP}~4.2  you  should
probably scan this section  *before*  you  read  Chapter~"Non-interactive
ANUPQ   functions".   If   the   troubleshooting   hints   don't    help,
Section~"Authors and Acknowledgements" below, gives contact  details  for
the authors of the components of the {\ANUPQ} package.

%%%%%%%%%%%%%%%%%%%%%%%%%%%%%%%%%%%%%%%%%%%%%%%%%%%%%%%%%%%%%%%%%%%%%%%%%
\Section{Authors and Acknowledgements}

The C implementation of the ANU `pq' standalone was developed by

\begintt
Eamonn O'Brien
Department of Mathematics
University of Auckland
Private Bag 92019
Auckland
New Zealand
\endtt
{\kernttindent}`email:' \Mailto{obrien@math.auckland.ac.nz}

The {\GAP} 4 version of this package was adapted from the {\GAP} 3
version by  

\begintt
Werner Nickel
AG 2, Fachbereich Mathematik, TU Darmstadt
Schlossgartenstr. 7, 64289 Darmstadt, Germany
\endtt
{\kernttindent}`email:' \Mailto{nickel@mathematik.tu-darmstadt.de}

An  interactive  interface  using  iostreams  was  developed   with   the
assistance of Werner Nickel by

%\begintt
\){\kernttindent}Greg Gamble
\){\kernttindent}Lehrstuhl D f\accent127ur Mathematik, RWTH Aachen
\){\kernttindent}Templergraben 64, 52062 Aachen, Germany
%\endtt

{\kernttindent}`email:' \Mailto{gregg@math.rwth-aachen.de} 

The authors would  like  to  thank  Joachim  Neub\accent127user  for  his
careful    proof-reading    and    advice,    and     for     formulating
Chapter~"Mathematical Background and Terminology".

We would also like to thank Bettina Eick who by her testing and provision
of examples helped us to eliminate a number of bugs and  who  provided  a
number of valuable suggestions for extensions of the package  beyond  the
{\GAP}~3 capabilities. 

\index{bug reports}
If you find a bug, the last section of {\ANUPQ}'s `README' describes  the
information we need and where to send us a bug report;  please  take  the
time to read this (i.e.~help us to help you).

%%%%%%%%%%%%%%%%%%%%%%%%%%%%%%%%%%%%%%%%%%%%%%%%%%%%%%%%%%%%%%%%%%%%%%%%%
\Section{Change history}

Below we list the main changes between versions of the {\ANUPQ} package.

\beginlist

\atindex{option pkgbanner}{@option \noexpand`pkgbanner'}
  
\item{`2.1'}
Binomial coefficient algorithm in `pq' program (now version 1.7) modified
to avoid overflow.

\item{}
A bug discovered by Tobias Ro{\ss}mann was fixed. It was  caused  in  the
{\ANUPQ} package's interface by not passing on the  parameter  `StepSize'
to the `pq' program properly. This bug  could  result  in  computing  the
wrong number of descendants if a `StepSize' different from 1 was chosen.

\item{}
Changes suggested by Gary Zablacki  were implemented in order to make the
package run under Windows with cygwin:

\itemitem{--}%unordered
use `[grp]' as default filename instead of `\<grp>',

\itemitem{--}
add target to the makefile for compiling the `pq'  program  with  cygwin,
and

\itemitem{--} 
run `PqQuitAll()' more often in test examples. This reduces the number of
simultaneously running processes and avoids bumping into system limits.

\item{`2.0'}
Changes to `pq' program (now version 1.6) and  {\GAP}  code  to  fix  bug
reported by Boris Girnat (thanks Boris!) where too many descendants  were
generated for a group of class more than 1  with  insoluble  automorphism
group. Also more changes for {\GAP}~4.4, for which

\itemitem{--}%unordered
option `pkgbanner' for suppression of the package  banner  is  deprecated
(now does nothing); its function is now provided by an optional  argument
of  the  `LoadPackage';  see  Section~"ref:LoadPackage"  in  the   {\GAP}
Reference Manual); and

\itemitem{--}
the package banner is no longer `Info'-ed at `InfoANUPQ' level 1, and  so
cannot be suppressed by setting the `InfoANUPQ' level to 0.

\item{}
For {\GAP}~4.3fix4, the previous behaviour  for  the  display/suppression
and `Info'-ing of the banner is unchanged, and `RequirePackage'  must  be
used rather than the new command `LoadPackage'.

\item{}
`PqSupplementInnerAutomorphisms' now returns a record  analogous  to  the
`AutomorphismGroupPGroup'  function  of  the  {\AutPGrp}   package,   and
`AutomorphismGroupPGroup' rather than `PqSupplementInnerAutomorphisms' is
now used to generate the required automorphism group data  in  descendant
calculations. The corresponding `PqExample' example  has  similarly  been
modified.

\item{`1.5'}
Fixed bug in the filtering of ones from input to `pq' program (thanks  to
Robert Morse again for  the  fix).  Some  preparatory  changes  made  for
{\GAP}~4.4.

\item{`1.4'}
Removed an unnecessary file and line from a file in the `src' directory.

\item{`1.3'}
Efficiency improvements: in particular, the use of `IsSyllableWordsFamily'
as first argument of some `FreeGroup' commands.

\item{`1.2'}
Fixed some inefficiencies in transmission of relators to the `pq' program
(thanks to  Robert  Morse  for  identifying  the  problem).  Updated  for
bugfixes in fix4 of {\GAP}~4.3, which required some usages of  `Pcgs'  to
be replaced by `GeneralizedPcgs'. Requires at least {\GAP}~4.3fix4 and an
updated {\AutPGrp} package that  uses  `SetGeneralizedPcgs'  in  lieu  of
`SetPcgs' in its definition of `ConvertHybridAutGroup'.

\item{`1.1'}
First {\GAP}~4 release. Requires  at  least  {\GAP}~4.2,  but  {\GAP}~4.3
recommended. Supersedes the earlier {\GAP}~3 version (1.0) in many  ways;
in particular, it provides an interactive interface  using  the  IOStream
technology introduced in {\GAP}~4.2.

\endlist

%%%%%%%%%%%%%%%%%%%%%%%%%%%%%%%%%%%%%%%%%%%%%%%%%%%%%%%%%%%%%%%%%%%%%%%%%
%%
%E
