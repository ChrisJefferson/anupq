%%%%%%%%%%%%%%%%%%%%%%%%%%%%%%%%%%%%%%%%%%%%%%%%%%%%%%%%%%%%%%%%%%%%%%%%%
%%
%W  basics.tex      ANUPQ documentation - background info   Werner Nickel
%W                                                      Joachim Neubueser
%%
%H  $Id$
%%
%%

%%%%%%%%%%%%%%%%%%%%%%%%%%%%%%%%%%%%%%%%%%%%%%%%%%%%%%%%%%%%%%%%%%%%%%%%%
\Chapter{Basics}

The {\GAP} 4 share package {\ANUPQ} provides an interface to the ANU PQ C
progam `pq' written by Eamonn O'Brien, making the functionality of the  C
program available to {\GAP}. Henceforth, we shall refer to  the  {\ANUPQ}
share package  (resp.  the  `pq'  binary),  when  discussing  the  {\GAP}
component (resp. the C code component) of ANU PQ. The ANU `pq' standalone
provides access to implementations of the following algorithms:

\beginlist

\item{1.}
A $p$-quotient algorithm to compute  power-commutator  presentations  for
$p$-factor groups of finitely presented groups. The algorithm implemented
here is based on that described by Newman and O'Brien \cite{NO96},  Havas
and Newman~\cite{HN80}, and papers referred to there. Another description
of  the   algorithm   is   given   by   Vaughan-Lee   (see~\cite{Vau90a},
\cite{Vau90b}). A FORTRAN implementation of this algorithm was programmed
by Alford and Havas. The basic data structures of that implementation are
retained.

\item{2.} 
A   $p$-group   generation   algorithm   to   generate   power-commutator
presentations of groups of prime power order. The  algorithm  implemented
here is based on the  algorithms  described  by  Newman~\cite{New77}  and
O'Brien~\cite{OBr90}. A FORTRAN  implementation  of  this  algorithm  was
developed earlier by Newman and O'Brien.

\item{3.}
A  standard  presentation  algorithm  used   to   compute   a   canonical
power-commutator presentation of a $p$-group. The  algorithm  implemented
here is described in~\cite{OBr94}.

\item{4.} 
An algorithm which can be used to compute the  automorphism  group  of  a
$p$-group. The algorithm implemented here is  described  in~\cite{OBr94}.
This part of  the  standalone  program  is  not  accessible  through  the
{\ANUPQ} share package.  Instead,  users  are  advised  to  consider  the
{\GAP}~4 share package {\AutPGrp}, which implements a better algorithm in
{\GAP} for the computation of automorphism groups of $p$-groups.

\endlist

Further   background   may   be   found   in~\cite{OBr95},   \cite{Vau84}
and~\cite{NNN98}.

For details regarding the standalone version see the file `guide.dvi'.

\index{capable}
We say a $p$-quotient descendant is *capable* if that  descendant  occurs
as the  factor  group  of  a  $p$-group  modulo  the  last  term  of  its
$p$-central series.

*confluent*

*pc presentation*

*exponent law*

*standard presentation*

*class*

*p-covering group*

*collection*

*equation ax = b*

*tails*

*descendants*

*exponent checks*

*maximal occurrences for pcp generators*

*compaction*

*echelonisation*

*automorphism action on generators*

*Engel (p-1)-identity*

*PAG-generating system/PcgsAutomorphisms*

*p-group generation*

*initial segment subgroup*

*rank of p-multiplicator*

*metabelian law*

*orbits*

*allowable subgroup*

*standard matrix*

*reduced p-covering group*

*immediate descendants*

*nuclear rank*

*definition sets*

*degree*

*label for standard matrix*

*vector space*

*isomorphism of standard presentation*

%%%%%%%%%%%%%%%%%%%%%%%%%%%%%%%%%%%%%%%%%%%%%%%%%%%%%%%%%%%%%%%%%%%%%%%%%
%%
%E
