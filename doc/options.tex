%%%%%%%%%%%%%%%%%%%%%%%%%%%%%%%%%%%%%%%%%%%%%%%%%%%%%%%%%%%%%%%%%%%%%%%%%%%%%
%%
%A  options.tex    ANUPQ documentation - options                Werner Nickel
%A                                                                Greg Gamble
%%
%A  @(#)$Id$
%%
%%

%%%%%%%%%%%%%%%%%%%%%%%%%%%%%%%%%%%%%%%%%%%%%%%%%%%%%%%%%%%%%%%%%%%%%%%%%
\Chapter{ANUPQ Options}

In this chapter we describe in detail all the options used  by  functions
of the {\ANUPQ} package. Note that by ``options'' we mean {\GAP}  options
that are passed to functions after the arguments and separated  from  the
arguments by a colon as described in Chapter~"ref:Function Calls" in  the
Reference Manual. The user is strongly advised to read Section~"Hints and
Warnings regarding the use of Options".

\>AllANUPQoptions() F

lists all the {\GAP}  options  defined  for  functions  of  the  {\ANUPQ}
package:

\beginexample
gap> AllANUPQoptions();
[ "AllDescendants", "BasicAlgorithm", "Bounds", "CapableDescendants", 
  "ClassBound", "CustomiseOutput", "Exponent", "Filename", "GroupName", 
  "Identities", "Metabelian", "NumberOfSolubleAutomorphisms", "OrderBound", 
  "OutputLevel", "PcgsAutomorphisms", "PqWorkspace", "Prime", 
  "PrintAutomorphisms", "PrintPermutations", "QueueFactor", 
  "RankInitialSegmentSubgroups", "RedoPcp", "RelativeOrders", "Relators", 
  "SetupFile", "SpaceEfficient", "StandardPresentationFile", "StepSize", 
  "SubList", "TreeDepth", "pQuotient" ]
\endexample

The following global variable gives a  partial  breakdown  of  where  the
above options are used.

\>`ANUPQoptions' V

is a record of lists of names of admissible {\ANUPQ} options, such  that
each field is either the name of a ``key'' {\ANUPQ} function  or  `other'
(for a miscellaneous list of functions) and the  corresponding  value  is
the list of option  names  that  are  admissible  for  the  function  (or
miscellaneous list of functions).

Also, from within a {\GAP} session, you may  use  {\GAP}'s  help  browser
(see Chapter~"ref:The Help System" in the {\GAP}  Reference  Manual);  to
find out about any particular {\ANUPQ} option,  simply  type:  ```?option
<option>''', where <option> is one of the options  listed  above  without
any quotes, e.g.

\begintt
gap> ?option Prime
\endtt

will display the sections  in  this  manual  that  describe  the  `Prime'
option. In fact the first 4 are for the functions that have `Prime' as an
option and the last actually  describes  the  option.  So  follow  up  by
choosing

\begintt
gap> ?5
\endtt

This is also the pattern for other options (the last section of the  list
always describes the option; the other sections are  the  functions  with
which the option may be used).

In the section following we describe in detail all {\ANUPQ}  options.  To
continue onto the next section on-line using {\GAP}'s help browser, type:

\begintt
gap> ?>
\endtt

%%%%%%%%%%%%%%%%%%%%%%%%%%%%%%%%%%%%%%%%%%%%%%%%%%%%%%%%%%%%%%%%%%%%%%%%%
\Section{Detailed descriptions of ANUPQ Options}

\beginitems

\>`Prime := <p>'{option Prime}@{option `Prime'|indexit}&
Specifies that the $p$-quotient for the prime  <p>  should  be  computed.

\>`ClassBound := <n>'{option  ClassBound}@{option  `ClassBound'|indexit}&
Specifies that the $p$-quotient to be  computed  has  lower  exponent-$p$
class at most <n>. If this option is omitted a default of  63  (which  is
the  maximum  possible  for  the  `pq'  program)  is  taken,  except  for
`PqDescendants' (see~"PqDescendants") and in a special case of `PqPCover'
(see~"PqPCover"). Let <F> be the argument (or start group of the  process
in the interactive case) for the function; then for  `PqDescendants'  the
default  is  `PClassPGroup(<F>)  +  1',  and  for  the  special  case  of
`PqPCover' the default is `PClassPGroup(<F>)'.

\>`pQuotient := <Q>'{option pQuotient}@{option `pQuotient'|indexit}&
This option is only available for the standard presentation functions. It
specifies that a $p$-quotient of the group argument of  the  function  or
group of the process is the pc <p>-group  <Q>,  where  <Q>  is  of  class
*less  than*  the  provided  (or  default)  value  of  `ClassBound'.   If
`pQuotient' is provided, then the option  `Prime'  if  also  provided, is
ignored; the prime <p> is discovered by computing `PrimePGroup(<Q>)'.

\>`Exponent := <n>'{option Exponent}@{option `Exponent'|indexit}&
Specifies that the $p$-quotient to be computed has exponent <n>.  For  an
interactive process, `Exponent' defaults to a previously  supplied  value
for the process. Otherwise (and non-interactively),  the  default  is  0,
which means that no exponent law is enforced.

\>`Relators := <rels>'{option Relators}@{option `Relators'|indexit}&
Specifies that the relators sent to the `pq'  program  should  be  <rels>
instead of the relators of the argument group <F> (or start group in  the
interactive case) of the calling function; <rels> should  be  a  list  of
*strings* in the string representations of the generators of <F>, and <F>
must be an *fp group* (even if the calling function accepts a pc  group).
This option provides a way  of  giving  relators  to  the  `pq'  program,
without having them pre-expanded by {\GAP}, which can sometimes effect  a
performance loss of the order of 100 (see Section~"The Relators Option").

*Notes*
\beginlist%ordered

\item{1.}
The `pq' program does not  use  `/'  to  indicate  multiplication  by  an
inverse and uses square brackets to represent (left normed)  commutators.
Also, even though the `pq' program accepts  relations,  all  elements  of
<rels> *must* be in relator form, i.e.~a relation of form `<w1>  =  <w2>'
must  be  written  as  `<w1>*(<w2>)^-1'  and  then  put  in  a  pair   of
double-quotes to make it a string. See the example below.

\item{2.}
To ensure there are no syntax errors in <rels>, each  relator  is  parsed
for validity via `PqParseWord' (see~"PqParseWord"). If  they  are  ok,  a
message to say so is `Info'-ed at `InfoANUPQ' level 2.

\endlist

\>`Metabelian'{option Metabelian}@{option `Metabelian'|indexit}&
Specifies that the largest metabelian $p$-quotient subject to  any  other
conditions specified by other options be  constructed.  By  default  this
restriction is not enforced.

\>`GroupName := <name>'{option GroupName}@{option `GroupName'|indexit}&
Specifies that the `pq' program should refer to the  group  by  the  name
<name> (a string). If `GroupName' is not  set  and  the  group  has  been
assigned a name via `SetName' (see~"ref:SetName") it is set as  the  name
the `pq' program should use. Otherwise, the ``generic''  name  `"\<grp>"'
is set as a default.

\>`Identities := <funcs>'{option Identities}@{option `Identities'|indexit}&
Specifies that the pc presentation should satisfy  the  laws  defined  by
each function in the list <funcs>. This option may  be  called  by  `Pq',
`PqEpimorphism', or `PqPCover' (see~"Pq").  Each  function  in  the  list
<funcs> must return a word in its arguments (there may be any  number  of
arguments). Let <identity> be one such function in <funcs>. Then as  each
lower exponent <p>-class quotient is formed, instances  $<identity>(<w1>,
\dots, <wn>)$ are added as relators to the pc presentation, where  $<w1>,
\dots, <wn>$ are words in the pc generators  of  the  quotient.  At  each
class the class and number of pc generators is `Info'-ed  at  `InfoANUPQ'
level 1, the number of instances is `Info'-ed at `InfoANUPQ' level 2, and
the instances that are evaluated are `Info'-ed at `InfoANUPQ' level 3. As
usual timing information is `Info'-ed at `InfoANUPQ' level 2; and details
of the processing of each instance from the `pq' program (which is  often
quite *voluminous*) is `Info'-ed at `InfoANUPQ' level 3. Try the examples
`"B2-4-Id"' and `"11gp-3-Engel-Id"' which demonstrate the  usage  of  the
`Identities' option; these are run using  `PqExample'  (see~"PqExample").
Take note of Note 1.~below in relation to the  example  `"B2-4-Id"';  the
companion example `"B2-4"' generates the same group using the  `Exponent'
option. These examples are discussed at length in Section~"The Identities
Option and PqEvaluateIdentities Function".

*Notes*

\beginlist%ordered

\item{1.}
Setting the `InfoANUPQ' level to 3 or more when setting the  `Identities'
option may slow down the computation considerably, by overloading  {\GAP}
with io operations.

\item{2.}
The `Identities' option is implemented at the {\GAP} level.  An  identity
that is just an exponent law should be  specified  using  the  `Exponent'
option (see~"option Exponent"), which is implemented at the C  level  and
is highly optimised and so is much more efficient.

\item{3.}
The  number  of   instances  of  each  identity   tends  to  grow
combinatorially with the class. So *care* should be exercised in using
the  `Identities'  option, by  including  other restrictions,  e.g.~by
using the `ClassBound' option (see~"option ClassBound").

\endlist

\>`OutputLevel := <n>'{option OutputLevel}@{option `OutputLevel'|indexit}&
Specifies the level of ``verbosity'' of the information output by the ANU
`pq' program when computing a pc presentation; <n> must be an integer  in
the range 0 to 3. `OutputLevel := 0' displays at most one line of  output
and is the default; `OutputLevel := 1' displays (usually)  slightly  more
output and `OutputLevel's of 2 and 3 are two levels of verbose output. To
see these messages from the `pq' program, the `InfoANUPQ' level  must  be
set to at least 1  (see~"InfoANUPQ").  See  Section~"Hints  and  Warnings
regarding the use of Options" for an example of how `OutputLevel' can  be
used as a troubleshooting tool.

\>`RedoPcp'{option RedoPcp}@{option `RedoPcp'|indexit}&
Specifies that the current pc presentation (for an  interactive  process)
stored by the `pq' program be scrapped  and  clears  the  current  values
stored for the options `Prime', `ClassBound', `Exponent' and `Metabelian'
and also clears the `pQuotient', `pQepi' and `pCover' fields of the  data
record of the process.

\>`SetupFile := <filename>'{option SetupFile}@{option `SetupFile'|indexit}&
Non-interactively, this option directs that `pq' should not be called and
that an input file  with  name  <filename>  (a  string),  containing  the
commands necessary for the  ANU  `pq'  standalone,  be  constructed.  The
commands written to <filename> are also `Info'-ed behind  a  ```ToPQ> '''
prompt at `InfoANUPQ' level  4  (see~"InfoANUPQ").  Except  in  the  case
following, the calling function returns `true'. If the  calling  function
is  the  non-interactive  version  of  one   of   `Pq',   `PqPCover'   or
`PqEpimorphism' and the group provided as argument is trivial given  with
an empty set of generators, then no setup file is written and  `fail'  is
returned (the `pq' program cannot do anything useful with such a  group).
Interactively, `SetupFile' is ignored.

*Note:*
Since commands emitted to the `pq' program may depend on knowing what the
``current state'' is, to form a setup file some ``close enough  guesses''
may sometimes be necessary; when this occurs a warning  is  `Info'-ed  at
`InfoANUPQ' or `InfoWarning' level 1. To determine  whether  the  ``close
enough guesses'' give an accurate setup file, it is necessary to run  the
command  without  the  `SetupFile'  option,  after  either  setting   the
`InfoANUPQ' level to at least 4  (the  setup  file  script  can  then  be
compared with the ```ToPQ> ''' commands that are `Info'-ed) or setting  a
`pq' command log file by using `ToPQLog' (see~"ToPQLog").

\>`PqWorkspace := <workspace>'{option PqWorkspace}@{option `PqWorkspace'|indexit}&
Non-interactively, this option sets the memory used by the `pq'  program.
It sets the maximum number of integer-sized elements to allocate  in  its
main storage array. By default, the `pq'  program  sets  this  figure  to
10000000. Interactively, `PqWorkspace' is ignored;  the  memory  used  in
this  case  may  be  set  by   giving   `PqStart'   a   second   argument
(see~"PqStart").

\>`PcgsAutomorphisms'{option PcgsAutomorphisms}%
@{option `PcgsAutomorphisms'|indexit}
\>`PcgsAutomorphisms := false'{option PcgsAutomorphisms}%
@{option `PcgsAutomorphisms'|indexit}&
Let <G> be the group associated with the calling function (or  associated
interactive process). Passing the option  `PcgsAutomorphisms'  without  a
value (or equivalently setting it to `true'), specifies that a polycyclic
generating sequence for the automorphism group (which must be  *soluble*)
of <G>, be computed and passed to the `pq'  program.  This  increases  the
efficiency of the computation; it also prevents  the  `pq'  from  calling
{\GAP} for orbit-stabilizer calculations. By default, `PcgsAutomorphisms'
is set to the value returned by `IsSolvable( AutomorphismGroup( <G> ) )',
and uses the package {\AutPGrp} to compute `AutomorphismGroup( <G> )'  if
it is installed. This flag is set to `true' or `false' in the  background
according  to  the  above  criterion  by  the  function   `PqDescendants'
(see~"PqDescendants" and~"PqDescendants!interactive").

*Note:*
If `PcgsAutomorphisms' is used when the  automorphism  group  of  <G>  is
insoluble, an error message occurs.

\>`OrderBound := <n>'{option OrderBound}@{option `OrderBound'|indexit}&
Specifies that only descendants of size at most $p^<n>$, where <n>  is  a
non-negative integer,  be  generated.  Note  that  you  cannot  set  both
`OrderBound' and `StepSize'.

\>`StepSize := <n>'{option StepSize}@{option `StepSize'}
\>`StepSize := <list>'{option StepSize}@{option `StepSize'|indexit}&
For  a  positive  integer  <n>,  `StepSize'  specifies  that  only  those
immediate descendants which are a factor $p^<n>$ bigger than their parent
group be generated.

&
For a list <list> of positive integers such that the sum of the length of
<list> and the exponent-$p$ class of <G> is  equal  to  the  class  bound
defined  by  the  option  `ClassBound',  `StepSize'  specifies  that  the
integers of <list> are the step sizes for each additional class.

\>`RankInitialSegmentSubgroups := <n>'{option RankInitialSegmentSubgroups}%
@{option `RankInitialSegmentSubgroups'|indexit}&
Sets the rank of the initial  segment  subgroup  chosen  to  be  <n>.  By
default, this has value 0.

\>`SpaceEfficient'{option SpaceEfficient}@{option `SpaceEfficient'|indexit}&
Specifies that the `pq' program performs certain calculations of $p$-group
generation more slowly  but with greater space efficiency.   This flag is
frequently  necessary for groups  of large  Frattini quotient  rank.  The
space saving  occurs because  only one permutation  is stored at  any one
time. This  option is only  available if the `PcgsAutomorphisms'  flag is
set  to  `true'  (see~"option  PcgsAutomorphisms").  For  an  interactive
process, `SpaceEfficient' defaults to a previously supplied value for the
process.   Otherwise  (and  non-interactively),  `SpaceEfficient'  is  by
default `false'.

\>`CapableDescendants'{option CapableDescendants}@{option `CapableDescendants'|indexit}&
By default, *all* (i.e.~capable and terminal) descendants  are  computed.
If this flag is set, only capable descendants are computed. Setting  this
option is equivalent to setting `AllDescendants  :=  false'  (see~"option
AllDescendants"),    except    if    both    `CapableDescendants'     and
`AllDescendants' are passed, `AllDescendants' is essentially ignored.
\goodbreak%

\>`AllDescendants := false'{option AllDescendants}@{option `AllDescendants'|indexit}&
By default, *all* descendants are constructed. If this  flag  is  set  to
`false', only capable descendants are computed. Passing  `AllDescendants'
without a value  (which  is  equivalent  to  setting  it  to  `true')  is
superfluous. This option is provided only for backward compatibility with
the  {\GAP}  3  version  of  the  {\ANUPQ}  package,  where  by   default
`AllDescendants'  was  set  to  `false'  (rather  than  `true').  It   is
preferable to use `CapableDescendants' (see~"option CapableDescendants").

\>`TreeDepth := <class>'{option TreeDepth}@{option `TreeDepth'|indexit}&
Specifies     that     the     descendants     tree     developed      by
`PqDescendantsTreeCoclassOne' (see~"PqDescendantsTreeCoclassOne")  should
be extended to class <class>, where <class> is a positive integer.

\>`SubList := <sub>'{option SubList}@{option `SubList'|indexit}&
Suppose that <L> is the list of descendants generated, then  for  a  list
<sub> of integers this option causes `PqDescendants' to return  `Sublist(
<L>, <sub> )'. If an integer <n>  is  supplied,  `PqDescendants'  returns
`<L>[<n>]'.

\>`NumberOfSolubleAutomorphisms := <n>'{option NumberOfSolubleAutomorphisms}%
@{option `NumberOfSolubleAutomorphisms'|indexit}&
Specifies that the number of soluble automorphisms  of  the  automorphism
group           supplied           by           `PqPGSupplyAutomorphisms'
(see~"PqPGSupplyAutomorphisms") in a $p$-group generation calculation  is
<n>. By default, <n> is taken to be  $0$;  <n>  must  be  a  non-negative
integer. If $<n> \ge 0$ then a  value  for  the  option  `RelativeOrders'
(see~"option RelativeOrders") must also be supplied.

\>`RelativeOrders := <list>'{option RelativeOrders}%
@{option `RelativeOrders'|indexit}&
Specifies the  relative  orders  of  each  soluble  automorphism  of  the
automorphism     group     supplied     by      `PqPGSupplyAutomorphisms'
(see~"PqPGSupplyAutomorphisms") in a  $p$-group  generation  calculation.
The list <list> must consist of <n> positive integers, where <n>  is  the
value   of   the   option   `NumberOfSolubleAutomorphisms'   (see~"option
NumberOfSolubleAutomorphisms"). By default <list> is empty.

\>`BasicAlgorithm'{option BasicAlgorithm}@{option `BasicAlgorithm'|indexit}&
Specifies that an  algorithm that the `pq' program  calls its ``default''
algorithm be  used for $p$-group generation. By default this algorithm is
*not*  used.   If  this  option  is  supplied  the  settings  of  options
`RankInitialSegmentSubgroups',     `AllDescendants',    `Exponent'    and
`Metabelian' are ignored.

\>`CustomiseOutput := <rec>'{option CustomiseOutput}%
@{option `CustomiseOutput'|indexit}&
Specifies that fine tuning of the output is  desired.  The  record  <rec>
should have any subset (or all) of the the following fields:

\quad`perm := <list>' &
where  <list>  is  a  list  of  booleans  which  determine  whether   the
permutation group output for the automorphism group should  contain:  the
degree, the extended automorphisms, the automorphism  matrices,  and  the
permutations, respectively.

\quad`orbit := <list>' &
where <list> is a list of booleans  which  determine  whether  the  orbit
output of the action of the automorphism group should contain: a summary,
and a complete listing of orbits, respectively. (It's  possible  to  have
*both* a summary and a complete listing.)

\quad`group := <list>' &
where <list> is a list of booleans  which  determine  whether  the  group
output should contain: the standard matrix of  each  allowable  subgroup,
the presentation of reduced  $p$-covering  groups,  the  presentation  of
immediate  descendants,  the  nuclear  rank  of  descendants,   and   the
$p$-multiplicator rank of descendants, respectively.

\quad`autgroup := <list>' &
where  <list>  is  a  list  of  booleans  which  determine  whether   the
automorphism group output should  contain:  the  commutator  matrix,  the
automorphism group description of descendants, and the automorphism group
order of descendants, respectively.

\quad`trace := <val>' &
where <val> is a boolean which if `true' specifies algorithm  trace  data
is desired. By default, one does not get algorithm trace data.

Not providing a field (or mis-spelling it!), specifies that  the  default
output is desired. As a convenience, `1' is also accepted as `true',  and
any value that is neither `1' nor `true' is taken as  `false'.  Also  for
each <list> above, an unbound list entry is taken as `false'.  Thus,  for
example

\begintt
CustomiseOutput := rec(group := [,,1], autgroup := [,1])
\endtt

specifies for the group output that only the  presentation  of  immediate
descendants is desired,  for  the  automorphism  group  output  only  the
automorphism group description of descendants  should  be  printed,  that
there should be no algorithm trace data,  and  that  the  default  output
should be provided for the permutation group and orbit output.

\>`StandardPresentationFile := <filename>'{option StandardPresentationFile}%
@{option `StandardPresentationFile'|indexit}&
Specifies that the file to which the standard presentation is written has
name <filename>. If the first character of the string <filename>  is  not
`/', <filename> is assumed to be the path of a writable file relative  to
the directory in which {\GAP} was started. If this option is  omitted  it
is written to the file with the name generated by the command  `Filename(
ANUPQData.tmpdir, "SPres" );', i.e.~the file with name `"SPres"'  in  the
temporary directory in which the `pq' program executes.

\>`QueueFactor := <n>'{option QueueFactor}@{option `QueueFactor'|indexit}&
Specifies a queue factor of <n>, where <n> should be a positive  integer.
This option may be used with `PqNextClass' (see~"PqNextClass").

The queue factor is used when the `pq' program uses automorphisms to close
a set of elements of the $p$-multiplicator under their action.

The algorithm used is a spinning algorithm:  it  starts  with  a  set  of
vectors in echelonized  form  (elements  of  the  $p$-multiplicator)  and
closes the span of these vectors under the action of  the  automorphisms.
For this each automorphism is applied to each vector and it is checked if
the result is contained in the span computed so far.  If  not,  the  span
becomes bigger and the vector is put into a queue and  the  automorphisms
are applied to this vector at a later stage. The process terminates  when
the automorphisms have been applied to all vectors  and  no  new  vectors
have been produced.

For each new vector it is decided, if its processing should  be  delayed.
If the vector contains too many non-zero entries, it is put into a second
queue. The elements in this queue are processed only when  there  are  no
elements in the first queue left.

The queue factor is a percentage figure. A vector is put into the  second
queue if the percentage of its non-zero entries exceeds the queue factor.

\>`Bounds := <list>'{option Bounds}@{option `Bounds'|indexit}&
Specifies a lower and upper bound on the indices of a list, where  <list>
is a pair of positive non-decreasing  integers.  See~"PqDisplayStructure"
and~"PqDisplayAutomorphisms" where this option may be used.

\>`PrintAutomorphisms := <list>'{option PrintAutomorphisms}%
@{option `PrintAutomorphisms'|indexit}&
Specifies that automorphism matrices be printed.

\>`PrintPermutations := <list>'{option PrintPermutations}%
@{option `PrintPermutations'|indexit}&
Specifies that permutations of the subgroups be printed.

\>`Filename := <string>'{option Filename}@{option `Filename'|indexit}&
Specifies that an output or input file to be written to or read  from  by
the `pq' program should have the name <string>.

\enditems

%%%%%%%%%%%%%%%%%%%%%%%%%%%%%%%%%%%%%%%%%%%%%%%%%%%%%%%%%%%%%%%%%%%%%%%%%
%%
%E
