\Section{ANU pq Package}

The   ANU  pq  provides  access  to  implementations   of  the  following
algorithms\:

1.  A $p$-quotient algorithm  to compute  a power-commutator presentation
for a group  of  prime power order.  The  algorithm implemented  here  is
based on that described in Newman and O\'Brien (1996), Havas and Newman 
(1980), and papers referred to there.   Another  description  of  the  
algorithm appears in  Vaughan-Lee (1990).  A FORTRAN implementation  of 
this algorithm  was  programmed  by Alford and Havas.  The basic  data 
structures  of that implementation are retained.

2. A $p$-group generation algorithm to generate descriptions of groups of
prime  power  order.   The  algorithm  implemented here is based  on  the
algorithms described in Newman  (1977)  and O\'Brien  (1990).  A  FORTRAN
implementation of  this  algorithm  was  earlier  developed by Newman and
O\'Brien.

3.   A  standard presentation   algorithm  used  to compute  a  canonical
power-commutator presentation  of a $p$-group. The  algorithm implemented
here is described in O\'Brien (1994).

4. An algorithm which can be used to compute  the automorphism group of a
$p$-group. The algorithm implemented here is described in O\'Brien (1994).

The following section describes the installation of the ANU pq package, a
description of the functions available in the  ANU pq package is given in
chapter "ANU Pq".

A reader interested  in details  of the  algorithms  and explanations  of
terms  used  is  referred  to  \cite{NOBr96}, \cite{HN80},  \cite{OBr90},  
\cite{OBr94}, \cite{OBr95},    \cite{New77},    \cite{Vau84},     
\cite{Vau90a},  and \cite{Vau90b}.

For  details about  the implementation and the standalone version see the
README. This implementation was developed in C by

Eamonn O\'Brien\\
Lehrstuhl D fuer Mathematik\\
RWTH Aachen

e-mail obrien@math.rwth-aachen.de

%%%%%%%%%%%%%%%%%%%%%%%%%%%%%%%%%%%%%%%%%%%%%%%%%%%%%%%%%%%%%%%%%%%%%%%%%%%%%
\Section{Installing the ANU pq Package}

The ANU  pq is written  in C and the  package can only be installed under
UNIX.  It has been tested on DECstation running Ultrix, a HP 9000/700 and
HP 9000/800 running HP-UX, a MIPS running RISC/os Berkeley, a NeXTstation
running NeXTSTEP 3.0, and SUNs running SunOS.

If you got  a complete binary  and source distribution for your  machine,
nothing has   to be done if  you  want to use   the ANU  pq for  a single
architecture. If  you want to use the  ANU pq for machines with different
architectures skip  the extraction and compilation  part of  this section
and proceed with the creation of shell scripts described below.

If you  got a complete source distribution,  skip the  extraction part of
this section and proceed with the compilation part below.

In the example we will assume that you, as user 'gap', are installing the
ANU pq package for use by several users on a network  of two DECstations,
called 'bert' and 'tiffy', and a NeXTstation,  called 'bjerun'. We assume
that  {\GAP}  is   also  installed  on   these  machines   following  the
instructions given in "Installation of GAP for UNIX".

Note that certain parts  of  the  output  in the examples should  only be
taken as rough outline, especially file sizes and file dates are *not* to
be taken literally.

First of  all you have to  get the file  'anupq.zoo' (see "Getting GAP").
Then you  must locate the {\GAP} directory  containing 'lib/' and 'doc/',
this is usually  'gap3r4p0' where '0'  is to be  replaced by  the current
patch level.

|    gap@tiffy:~ > ls -l
    drwxr-xr-x  11 gap     1024 Nov  8  1991 gap3r4p0
    -rw-r--r--   1 gap   360891 Dec 27 15:16 anupq.zoo
    gap@tiffy:~ > ls -l gap3r4p0
    drwxr-xr-x   2 gap     3072 Nov 26 11:53 doc
    drwxr-xr-x   2 gap     1024 Nov  8  1991 grp
    drwxr-xr-x   2 gap     2048 Nov 26 09:42 lib
    drwxr-xr-x   2 gap     2048 Nov 26 09:42 pkg
    drwxr-xr-x   2 gap     2048 Nov 26 09:42 src
    drwxr-xr-x   2 gap     1024 Nov 26 09:42 tst|

Unpack the package using 'unzoo'  (see "Installation  of GAP for  UNIX").
Note that  you must be in the  directory containing 'gap3r4p0'  to unpack
the files.    After you have  unpacked   the source  you may  remove  the
*archive-file*.

|    gap@tiffy:~ > unzoo x anupq
    gap@tiffy:~ > cd gap3r4p0/pkg/anupq
    gap@tiffy:../anupq> ls -l
    drwxr-xr-x  5 gap       512 Feb 24 11:17 MakeLibrary
    -rw-r--r--  1 gap     28926 Jun  8 14:21 Makefile
    -rw-r--r--  1 gap      8818 Jun  8 14:21 README
    -rw-r--r--  1 gap       753 Jun 23 18:59 StandardPres
    drwxr-xr-x  2 gap      1024 Jun  8 14:15 TEST
    drwxr-xr-x  2 gap       512 Jun 16 16:03 bin
    drwxr-xr-x  2 gap       512 May 16 06:58 cayley
    drwxr-xr-x  2 gap       512 Jun  8 08:48 doc
    drwxr-xr-x  2 gap      1024 Mar  5 04:01 examples
    drwxr-xr-x  2 gap       512 Jun 23 16:37 gap
    drwxr-xr-x  2 gap       512 Jun 24 10:51 include
    -rw-rw-rw-  1 gap       867 Jun  9 16:12 init.g
    drwxr-xr-x  2 gap      1024 May 21 02:28 isom
    drwxr-xr-x  2 gap       512 May 16 07:58 magma
    drwxr-xr-x  2 gap      6656 Jun 24 11:10 src |

Typing 'make' will produce a list of possible target.

|gap@tiffy:../anupq > make
usage: 'make <target> EXT=<ext>'  where <target> is one of
'dec-mips-ultrix-gcc2-gmp' for DECstations under Ultrix with gcc/gmp
'dec-mips-ultrix-cc-gmp'   for DECstations under Ultrix with cc/gmp
'dec-mips-ultrix-gcc2'     for DECstations under Ultrix with gcc
'dec-mips-ultrix-cc'       for DECstations under Ultrix with cc
'hp-hppa1.1-hpux-cc-gmp'   for HP 9000/700 under HP-UX with cc/gmp
'hp-hppa1.1-hpux-cc'       for HP 9000/700 under HP-UX with cc
'hp-hppa1.0-hpux-cc-gmp'   for HP 9000/800 under HP-UX with cc/gmp
'hp-hppa1.0-hpux-cc'       for HP 9000/800 under HP-UX with cc
'ibm-i386-386bsd-gcc2-gmp' for IBM PCs under 386BSD with gcc/gmp
'ibm-i386-386bsd-cc-gmp'   for IBM PCs under 386BSD with cc/gmp
'ibm-i386-386bsd-gcc2'     for IBM PCs under 386BSD with gcc2
'ibm-i386-386bsd-cc'       for IBM PCs under 386BSD with cc
'mips-mips-bsd-cc-gmp'     for MIPS under RISC/os Berkeley with cc/gmp
'mips-mips-bsd-cc'         for MIPS under RISC/os Berkeley with cc
'next-m68k-mach-gcc2-gmp'  for NeXT under Mach with gcc/gmp
'next-m68k-mach-cc-gmp'    for NeXT under Mach with cc/gmp
'next-m68k-mach-gcc2'      for NeXT under Mach with gcc
'next-m68k-mach-cc'        for NeXT under Mach with cc
'sun-sparc-sunos-gcc2-gmp' for SUN 4 under SunOs with gcc/gmp
'sun-sparc-sunos-cc-gmp'   for SUN 4 under SunOs with cc/gmp
'sun-sparc-sunos-gcc2'     for SUN 4 under SunOs with gcc2
'sun-sparc-sunos-cc'       for SUN 4 under SunOs with cc
'unix-gmp'                 for a generic unix system with cc/gmp
'unix'                     for a generic unix system with cc
'clean'                    remove all created files

   where <ext> should be a sensible extension, i.e.,
   'EXT=-sun-sparc-sunos' for SUN 4 or 'EXT=' if the PQ only
   runs on a single architecture

   targets are listed according to preference,
   i.e., 'sun-sparc-sunos-gcc2' is better than 'sun-sparc-sunos-cc'.
   additional C compiler and linker flags can be passed with
   'make <target> COPTS=<compiler-opts> LOPTS=<linker-opts>',
   i.e., 'make sun-sparc-sunos-cc COPTS=-g LOPTS=-g'.

   set GAP if gap 3.4 is not started with the command 'gap',
   i.e., 'make sun-sparc-sunos-cc GAP=/home/gap/bin/gap-3.4'.

   in order to use the GNU multiple precision (gmp) set
   'GNUINC' (default '/usr/local/include') and 
   'GNULIB' (default '/usr/local/lib')|

Select the  target you need.  If  you  have the *GNU*  multiple precision
arithmetic (gmp)  installed on your  system, select the target  ending in
'-gmp'. Note that  the  gmp is  *not  required*.   In our  case we  first
compile   the DECstation version.   We assume  that  the command to start
{\GAP}    is  '/usr/local/bin/gap'   for   'tiffy'    and  'bjerun'   and
'/rem/tiffy/usr/local/bin/gap' for 'bert'.

|    gap@tiffy:../anupq > make dec-mips-ultric-cc \
                               GAP=/usr/local/bin/gap \
			       EXT=-dec-mips-ultrix
    # you will see a lot of messages and a few warnings |

Now repeat the compilation  for the NeXTstation. *Do not* forget to clean
up.

|    gap@tiffy:../anupq > rlogin bjerun
    gap@bjerun:~ > cd gap3r4p0/pkg/anupq
    gap@bjerun:../anupq > make clean
    gap@bjerun:../src > make next-m68k-mach-cc \
                             GAP=/usr/local/bin/gap \
			     EXT=-next-m68k-mach
    # you will see a lot of messages and a few warnings
    gap@bjerun:../anupq > exit
    gap@tiffy:../anupq > |

Switch into the subdirectory 'bin/'  and create a  script which will call
the correct binary for each machine. A skeleton  shell script is provided
in 'bin/pq.sh'.

|    gap@tiffy:../anupq > cd bin
    gap@tiffy:../bin > cat > pq
    |\#|!/bin/csh
    switch ( `hostname` )
      case 'tiffy':
        exec $0-dec-mips-ultrix $* ;
        breaksw ;
      case 'bert':
        setenv ANUPQ_GAP_EXEC /rem/tiffy/usr/local/bin/gap ;
        exec $0-dec-mips-ultrix $* ;
        breaksw ;
      case 'bjerun':
        limit stacksize 2048 ;
        exec $0-next-m68k-mach $* ;
        breaksw ;
      default:
        echo "pq: sorry, no executable exists for this machine" ;
        breaksw ;
    endsw
    |<ctr>-'D'|
    gap@tiffy:../bin > chmod 755 pq
    gap@tiffy:../bin > cd .. |

Note that the  NeXTstation requires you to raise the stacksize.   If your
default limit on any other machine for the stack size  is less than  1024
you might need to add the 'limit stacksize 2048' line.
     
If the documentation is  not  already installed  or an older  version  is
installed, copy  the file 'gap/anupq.tex' into  the  'doc/' directory and
run latex  again (see "Installation  of  GAP for  UNIX").  In general the
documentation will   already  be installed  so  you   can  just skip  the
following step.

|    gap@tiffy:../anupq > cp gap/anupq.tex ../../doc
    gap@tiffy:../anupq > cd ../../doc
    gap@tiffy:../doc > latex manual
    # a few messages about undefined references
    gap@tiffy:../doc > latex manual
    # a few messages about undefined references
    gap@tiffy:../doc > makeindex manual
    # 'makeindex' prints some diagnostic output
    gap@tiffy:../doc > latex manual
    # there should be no warnings this time
    gap@tiffy:../doc cd ../pkg/anupq |

Now it is time to test  the installation.  The  first test will only test
the ANU pq.

|    gap@tiffy:../anupq > bin/pq < gap/test1.pga
    # a lot of messages ending in
    **************************************************
    Starting group: c3c3 |\#|2;2 |\#|4;3
    Order: 3^7
    Nuclear rank: 3
    3-multiplicator rank: 4
    |\#| of immediate descendants of order 3^8 is 7
    |\#| of capable immediate descendants is 5

    **************************************************
    34 capable groups saved on file c3c3_class4
    Construction of descendants took 1.92 seconds

    Select option: 0 
    Exiting from p-group generation

    Select option: 0 
    Exiting from ANU p-Quotient Program
    Total user time in seconds is 1.97
    gap@tiffy:../anupq > ls -l c3c3*
    total 89
    -rw-r--r--    1 gap    3320 Jun 24 11:24 c3c3_class2
    -rw-r--r--    1 gap    5912 Jun 24 11:24 c3c3_class3
    -rw-r--r--    1 gap   56184 Jun 24 11:24 c3c3_class4
    gap:../anupq > rm c3c3_class* |

The second test will test the stacksize. If it is too small you will  get
a memory fault, try to raise the stacksize as described above.

|    gap@tiffy:../anupq > bin/pq < gap/test2.pga
    # a lot of messages ending in
    **************************************************
    Starting group: c2c2 |\#|1;1 |\#|1;1 |\#|1;1
    Order: 2^5
    Nuclear rank: 1
    2-multiplicator rank: 3
    Group c2c2 |\#|1;1 |\#|1;1 |\#|1;1 is an invalid starting group

    **************************************************
    Starting group: c2c2 |\#|2;1 |\#|1;1 |\#|1;1
    Order: 2^5
    Nuclear rank: 1
    2-multiplicator rank: 3
    Group c2c2 |\#|2;1 |\#|1;1 |\#|1;1 is an invalid starting group
    Construction of descendants took 0.47 seconds

    Select option: 0 
    Exiting from p-group generation

    Select option: 0 
    Exiting from ANU p-Quotient Program
    Total user time in seconds is 0.50
    gap@tiffy:../anupq > ls -l c2c2*
    total 45
    -rw-r--r--    1 gap   6228 Jun 24 11:25 c2c2_class2
    -rw-r--r--    1 gap  11156 Jun 24 11:25 c2c2_class3
    -rw-r--r--    1 gap   2248 Jun 24 11:25 c2c2_class4
    -rw-r--r--    1 gap      0 Jun 24 11:25 c2c2_class5
    gap:../anupq > rm c2c2_class* |

The third example tests the link between the ANU pq and {\GAP}.  If there
is a problem you will get a error message saying
'Error in  system  call  to GAP';  if this happens, check the environment
variable 'ANUPQ\_GAP\_EXEC'.

|    gap@tiffy:../anupq > bin/pq < gap/test3.pga
    # a lot of messages ending in
    **************************************************
    Starting group: c5c5 |\#|1;1 |\#|1;1
    Order: 5^4
    Nuclear rank: 1
    5-multiplicator rank: 2
    |\#| of immediate descendants of order 5^5 is 2

    **************************************************
    Starting group: c5c5 |\#|1;1 |\#|2;2
    Order: 5^5
    Nuclear rank: 3
    5-multiplicator rank: 3
    |\#| of immediate descendants of order 5^6 is 3
    |\#| of immediate descendants of order 5^7 is 3
    |\#| of capable immediate descendants is 1
    |\#| of immediate descendants of order 5^8 is 1
    |\#| of capable immediate descendants is 1

    **************************************************
    2 capable groups saved on file c5c5_class4

    **************************************************
    Starting group: c5c5 |\#|1;1 |\#|2;2 |\#|4;2
    Order: 5^7
    Nuclear rank: 1
    5-multiplicator rank: 2
    |\#| of immediate descendants of order 5^8 is 2
    |\#| of capable immediate descendants is 2

    **************************************************
    Starting group: c5c5 |\#|1;1 |\#|2;2 |\#|7;3
    Order: 5^8
    Nuclear rank: 2
    |\#| of immediate descendants of order 5^9 is 1
    |\#| of capable immediate descendants is 1
    |\#| of immediate descendants of order 5^10 is 1
    |\#| of capable immediate descendants is 1

    **************************************************
    4 capable groups saved on file c5c5_class5
    Construction of descendants took 0.62 seconds

    Select option: 0 
    Exiting from p-group generation

    Select option: 0 
    Exiting from ANU p-Quotient Program
    Total user time in seconds is 0.68
    gap@tiffy:../anupq > ls -l c5c5*
    total 41
    -rw-r--r--    1 gap     924 Jun 24 11:27 c5c5_class2
    -rw-r--r--    1 gap    2220 Jun 24 11:28 c5c5_class3
    -rw-r--r--    1 gap    3192 Jun 24 11:30 c5c5_class4
    -rw-r--r--    1 gap    7476 Jun 24 11:32 c5c5_class5
    gap:../anupq > rm c5c5_class* |

The fourth test will test the standard presentation part of the pq.

|    gap@tiffy:../anupq > bin/pq -i -k < gap/test4.sp
    # a lot of messages ending in
    Computing standard presentation for class 9 took 0.43 seconds
    The largest 5-quotient of the group has class 9

    Select option: 0 
    Exiting from ANU p-Quotient Program
    Total user time in seconds is 2.17
    gap@tiffy:../anupq > ls -l SPRES
    -rw-r--r--  1 gap     768 Jun 24 11:33 SPRES
    gap@tiffy:../anupq > diff SPRES gap/out4.sp
    # there should be no difference if compiled with '-gmp'
    156250000
    gap@tiffy:../anupq > rm SPRES |

The last  test  will test the  link  between {\GAP} and   the ANU pq.  If
everything goes well you should not see any message.

|    gap@tiffy:../anupq > gap -b
    gap> RequirePackage( "anupq" );
    gap> ReadTest( "gap/anupga.tst" );
    gap> |

You may now repeat the tests for the other machines.

