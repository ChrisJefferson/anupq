%%%%%%%%%%%%%%%%%%%%%%%%%%%%%%%%%%%%%%%%%%%%%%%%%%%%%%%%%%%%%%%%%%%%%%%%%%%%%
%%
%A  anupq.tex    ANUPQ documentation - non-interactive f'ns    Eamonn O'Brien
%A                                                             & Frank Celler
%A                                                           & Benedikt Rothe
%%
%A  @(#)$Id$
%%
%%

%%%%%%%%%%%%%%%%%%%%%%%%%%%%%%%%%%%%%%%%%%%%%%%%%%%%%%%%%%%%%%%%%%%%%%%%%
\Chapter{Non-interactive ANUPQ functions}

Here we describe  all  the  non-interactive  functions  of  the  {\ANUPQ}
package; i.e.~``one-shot'' functions that invoke the `pq' binary in  such
a way that once {\GAP} has got what it needs, the `pq' binary is  allowed
to exit. It is expected that most of the time users will only need  these
functions. The functions interface with three of the four algorithms (see
Chapter~"Introduction") provided by the  ANU  `pq'  C  program,  and  are
mainly grouped according to the algorithm of the `pq' binary they  relate
to.

In Section~"Computing p-Quotients", we describe the functions  that  give
access to the $p$-quotient algorithm.

Section~"Computing Standard Presentations" describe functions  that  give
access to the standard presentation algorithm.

Section~"Testing  p-Groups  for  Isomorphism"  describe  functions   that
implement  an  isomorphism  test  for  $p$-groups  using   the   standard
presentation algorithm.

In Section~"Computing Descendants of a p-Group",  we  describe  functions
that give access to the $p$-group generation algorithm.

To use any of the functions one must have at some stage previously typed:

\beginexample
gap> RequirePackage("anupq");
\endexample

(the response of which we have omitted; see~"Loading the ANUPQ Package").
We will nevertheless remind you of this for each function,  in  case  you
happen to be reading about that particular function via on-line help.

%%%%%%%%%%%%%%%%%%%%%%%%%%%%%%%%%%%%%%%%%%%%%%%%%%%%%%%%%%%%%%%%%%%%%%%%%
\Section{Computing  p-Quotients}

\>Pq( <F> : <options> ) F

returns for the fp or pc group <F>, the $p$-quotient of <F> specified  by
<options>, as a pc group. Following the colon, <options> is  a  selection
of the options from the following list, separated by commas  like  record
components (see Section~"ref:function call with options"  in  the  {\GAP}
Reference Manual). As a minimum the user *must* supply a  value  for  the
`Prime' option. Below  we  list  the  options  recognised  by  `Pq'  (see
Chapter~"ANUPQ options" for detailed descriptions).

\beginlist%unordered

\atindex{option Prime}{@option \noexpand`Prime'}
\item{}`Prime := <p>'

\atindex{option ClassBound}{@option \noexpand`ClassBound'}
\item{}`ClassBound := <n>'

\atindex{option Exponent}{@option \noexpand`Exponent'}
\item{}`Exponent := <n>'

\atindex{option Relators}{@option \noexpand`Relators'}
\item{}`Relators := <rels>'

\atindex{option Metabelian}{@option \noexpand`Metabelian'}
\item{}`Metabelian'

\atindex{option GroupName}{@option \noexpand`GroupName'}
\item{}`GroupName := <name>'

\atindex{option OutputLevel}{@option \noexpand`OutputLevel'}
\item{}`OutputLevel := <n>'

\atindex{option SetupFile}{@option \noexpand`SetupFile'}
\item{}`SetupFile := <filename>'

\atindex{option PqWorkspace}{@option \noexpand`PqWorkspace'}
\item{}`PqWorkspace := <workspace>'

\endlist

*Notes:* `Pq' may also  be  called  with  no  arguments  or  one  integer
argument,   in   which   case   it   is    being    used    interactively
(see~"Pq!interactive");  the  same  options  may  be  used,  except  that
`SetupFile'  and  `PqWorkspace'  are  ignored  by  the  interactive  `Pq'
function.

See Section~"Attributes and a Property for fp and pc  p-groups"  for  the
attributes   and   property   `NuclearRank',   `MultiplicatorRank'    and
`IsCapable' which may be applied to the group returned by `Pq'.

(`Pq' requires the {\ANUPQ} package; see~"Loading the ANUPQ Package".)

See also `PqEpimorphism' ("PqEpimorphism").

We now give a few examples of the use of `Pq'. Except for the addition of
a few comments and the non-suppression of output (by not using duplicated
semicolons) the same examples may be run by typing: `PqExample( "Pq"  );'
(see~"PqExample").

\beginexample
gap> RequirePackage("anupq");; # does nothing if ANUPQ is already loaded
gap> # First we get a p-quotient of a free group of rank 2
gap> F := FreeGroup("a", "b");; a := F.1;; b := F.2;;
gap> Pq( F : Prime := 2, ClassBound := 3 ); 
<pc group of size 1024 with 10 generators>
gap> # Now let us get a p-quotient of an fp group
gap> G := F / [a^4, b^4];
<fp group on the generators [ a, b ]>
gap> Pq( G : Prime := 2, ClassBound := 3 ); 
<pc group of size 256 with 8 generators>
gap> # Now let's get a different p-quotient of the same group
gap> Pq( G : Prime := 2, ClassBound := 3, Exponent := 4); 
<pc group of size 128 with 7 generators>
gap> # Now we'll get a p-quotient of another fp group
gap> # which we will redo using the `Relators' option
gap> R := [ a^25, Comm(Comm(b, a), a), b^5 ];
[ a^25, a^-1*b^-1*a*b*a^-1*b^-1*a^-1*b*a^2, b^5 ]
gap> H := F / R;
<fp group on the generators [ a, b ]>
gap> Pq( H : Prime := 5, ClassBound := 5, Metabelian);
<pc group of size 78125 with 7 generators>
\endexample

\atindex{option Relators!example of usage}%
{@option \noexpand`Relators'!example of usage}
Now we redo the last example to show  how  one  may  use  the  `Relators'
option. Observe that `Comm(Comm(b, a), a)' is a  left  normed  commutator
which must be written in square bracket notation for the `pq' binary  and
embedded in  a  pair  of  double  quotes.  The  function  `PqGAPRelators'
(see~"PqGAPRelators") can be used to translate a list of strings prepared
for the `Relators' option into {\GAP} format. Below we  use  it.  Observe
that the value of `R' is the same as before.

\beginexample
gap> F := FreeGroup("a", "b");;
gap> # `F' was defined for `Relators'. We use the same strings that GAP uses
gap> # for printing the free group generators. It is *not* necessary to
gap> # predefine: a := F.1; etc. (as it was above).
gap> rels := [ "a^25", "[b, a, a]", "b^5" ];
[ "a^25", "[b, a, a]", "b^5" ]
gap> R := PqGAPRelators(F, rels);
[ a^25, a^-1*b^-1*a*b*a^-1*b^-1*a^-1*b*a^2, b^5 ]
gap> H := F / R;
<fp group on the generators [ a, b ]>
gap> Pq( H : Prime := 5, ClassBound := 5, Metabelian, Relators := rels);
<pc group of size 78125 with 7 generators>
\endexample

In fact, above we could have just passed `F' (rather than  `H'),  i.e.~we
could have done:

\beginexample
gap> F := FreeGroup("a", "b");;
gap> rels := [ "a^25", "[b, a, a]", "b^5" ];
[ "a^25", "[b, a, a]", "b^5" ]
gap> Pq( F : Prime := 5, ClassBound := 5, Metabelian, Relators := rels);
<pc group of size 78125 with 7 generators>
\endexample

The non-interactive `Pq' function also allows the options to be passed in
two other ways; these alternatives have been included for those  familiar
with the {\GAP}~3 version of the {\ANUPQ} package; the  preferred  method
of passing options is the one already described.  Firstly,  they  may  be
passed in a record as a second argument; note that  any  boolean  options
must be set explicitly e.g.

\beginexample
gap> Pq( H, rec( Prime := 5, ClassBound := 5, Metabelian := true ) );
<pc group of size 78125 with 7 generators>
\endexample

It is also possible to pass them as extra arguments,  where  each  option
name appears as a string followed immediately by  its  value  (if  not  a
boolean option) e.g.

\beginexample
gap> Pq( H, "Prime", 5, "ClassBound", 5, "Metabelian" );             
<pc group of size 78125 with 7 generators>
\endexample

This method of passing options permits abbreviation; the only restriction
is that the abbreviation must be  unique.  So  `"Pr"'  may  be  used  for
`"Prime"', `"Class"' or even just `"C"' for `"ClassBound"', etc.

The above examples can be run  from  {\GAP}  via  `PqExample( "Pq-ni" );'
(see~"PqExample").

\>PqEpimorphism( <F> : <options> ) F

returns for the fp or pc group <F>  an  epimorphism  from  <F>  onto  the
$p$-quotient  of  <F>  specified  by  <options>;  the  possible   options
<options> and *required* option (`"Prime"') are as for  `Pq'  (see~"Pq").
`PqEpimorphism' only differs from `Pq' in  what  it  outputs;  everything
about what must/may be passed as input to `PqEpimorphism' is the same  as
for `Pq'.  The  same  alternative  methods  of  passing  options  to  the
non-interactive  `Pq'  function  are  available  to  the  non-interactive
version of `PqEpimorphism'.

\beginexample
gap> F := FreeGroup (2, "F");
<free group on the generators [ F1, F2 ]>
gap> phi := PqEpimorphism( F : Prime := 5, ClassBound := 2 );
[ F1, F2 ] -> [ f1, f2 ]
gap> Image( phi );
<pc group of size 3125 with 5 generators>
\endexample

Typing: `PqExample( "PqEpimorphism" );' runs the above example in  {\GAP}
(see~"PqExample").

*Notes:* `PqEpimorphism' may also be called  with  no  arguments  or  one
integer  argument,  in  which  case  it  is  being   used   interactively
(see~"PqEpimorphism!interactive"),  and  the  options   `SetupFile'   and
`PqWorkspace' are ignored by the interactive `PqEpimorphism' function.

See Section~"Attributes and a Property for fp and pc  p-groups"  for  the
attributes   and   property   `NuclearRank',   `MultiplicatorRank'    and
`IsCapable' which may be applied to the image group  of  the  epimorphism
returned by `PqEpimorphism'.

\>PqPCover( <F> : <options> ) F

returns for the fp or  pc  group  <F>,  the  $p$-covering  group  of  the
$p$-quotient of <F> specified by  <options>,  as  a  pc  group,  i.e.~the
$p$-covering group of the $p$-quotient `Pq( <F> : <options> )'. Thus  the
options that `PqPCover' accepts are exactly those expected for `Pq'  (and
hence as a minimum the user *must* supply a value for the `Prime' option;
see~"Pq" for more details), except in the following special case.

If <F> is already a $p$-group, in the sense  that  `HasIsPGroup(<F>)  and
IsPGroup(<F>)' is `true', then

\beginitems

`Prime'& 
defaults to `PrimePGroup(<F>)', if not supplied and  `HasPrimePGroup(<F>)
= true'; and

`ClassBound'&
defaults to `PrimePGroup(<F>)' if `HasPClassPGroup(<F>) =  true'  if  not
supplied, or to the usual default of 63, otherwise.

\enditems

The same alternative methods of passing options  to  the  non-interactive
`Pq' function are available to the non-interactive version of `PqPCover'.

We now give a few examples of the use of `PqPCover'. These  examples  are
just the same as the ones we gave for `Pq'  (see~"Pq"),  except  that  in
each instance  the  command  `Pq'  has  been  replaced  with  `PqPCover'.
Essentially  the  same  examples  may  be  run  by  typing:   `PqExample(
"PqPCover" );' (see~"PqExample").

\beginexample
gap> F := FreeGroup("a", "b");; a := F.1;; b := F.2;;
gap> PqPCover( F : Prime := 2, ClassBound := 3 );
<pc group of size 262144 with 18 generators>
gap> 
gap> # Now let's get a p-cover of a p-quotient of an fp group
gap> G := F / [a^4, b^4];
<fp group on the generators [ a, b ]>
gap> PqPCover( G : Prime := 2, ClassBound := 3 );
<pc group of size 16384 with 14 generators>
gap> 
gap> # Now let's get a p-cover of a different p-quotient of the same group
gap> PqPCover( G : Prime := 2, ClassBound := 3, Exponent := 4);
<pc group of size 8192 with 13 generators>
gap> 
gap> # Now we'll get a p-cover of a p-quotient of another fp group
gap> # which we will redo using the `Relators' option
gap> R := [ a^25, Comm(Comm(b, a), a), b^5 ];
[ a^25, a^-1*b^-1*a*b*a^-1*b^-1*a^-1*b*a^2, b^5 ]
gap> H := F / R;
<fp group on the generators [ a, b ]>
gap> PqPCover( H : Prime := 5, ClassBound := 5, Metabelian);
<pc group of size 48828125 with 11 generators>
gap> 
gap> # Now we redo the previous example using the `Relators' option
gap> F := FreeGroup("a", "b");;
gap> # `F' was defined for `Relators'. We use the same strings that GAP uses
gap> # for printing the free group generators. It is *not* necessary to
gap> # predefine: a := F.1; etc. (as it was above).
gap> rels := [ "a^25", "[b, a, a]", "b^5" ];
[ "a^25", "[b, a, a]", "b^5" ]
gap> R := PqGAPRelators(F, rels);
[ a^25, a^-1*b^-1*a*b*a^-1*b^-1*a^-1*b*a^2, b^5 ]
gap> H := F / R;
<fp group on the generators [ a, b ]>
gap> PqPCover( H : Prime := 5, ClassBound := 5, Metabelian, Relators := rels);
<pc group of size 48828125 with 11 generators>
gap> 
gap> # Above we could have just passed `F' (rather than `H'):
gap> F := FreeGroup("a", "b");
<free group on the generators [ a, b ]>
gap> rels := [ "a^25", "[b, a, a]", "b^5" ];
[ "a^25", "[b, a, a]", "b^5" ]
gap> PqPCover( F : Prime := 5, ClassBound := 5, Metabelian, Relators := rels);
<pc group of size 48828125 with 11 generators>
\endexample

%%%%%%%%%%%%%%%%%%%%%%%%%%%%%%%%%%%%%%%%%%%%%%%%%%%%%%%%%%%%%%%%%%%%%%%%%
\Section{Computing Standard Presentations}

\index{automorphisms!of $p$-groups}
\>PqStandardPresentation( <F>, <p> : <options> ) F
\>PqStandardPresentation( <F>, <Q> : <options> ) F
\>StandardPresentation( <F>, <p> : <options> ) M
\>StandardPresentation( <F>, <Q> : <options> ) M

Here <F> is an fp or pc group and the second argument is either  a  prime
<p> or a pc <p>-group <Q>. If the user supplies a pc group <Q> as  second
argument which is a $p$-quotient of <F>, the package {\AutPGrp} is called
automatically to compute the automorphism group of  <Q>;  an  error  will
occur that asks the  user  to  install  the  package  {\AutPGrp}  if  the
automorphism  group  cannot  be  computed.  The  functions   return   the
<p>-quotient of <F>, specified by the second argument and  <options>,  as
an fp group which has a standard presentation.

Following the colon, <options> is a selection of  the  options  from  the
following  list,  separated  by  commas  like  record   components   (see
Section~"ref:function call with options" in the {\GAP} Reference Manual).
Section~"Hints and Warnings regarding the  use  of  Options"  gives  some
important hints and warnings regarding option usage.

*Notes:*
In contrast to the function `Pq' (see~"Pq") which  returns  a  pc  group,
here an fp group is returned. This is because the output is  mainly  used
for isomorphism testing for which an fp group  is  enough.  However,  the
presentation is essentially a polycyclic presentation and if you need  to
do any further computation with this group (e.g.~to find the  order)  you
can use the function `PcGroupFpGroup'  (see~"ref:PcGroupFpGroup"  in  the
{\GAP} Reference Manual) to form a pc group.

The  attributes  and  property  `NuclearRank',  `MultiplicatorRank'   and
`IsCapable' are set for the group returned by `PqStandardPresentation' or
`StandardPresentation' (see Section~"Attributes and a Property for fp and
pc p-groups").

The following options  are  recognised  by  `PqStandardPresentation'  and
`StandardPresentation'  (see   Chapter~"ANUPQ   options"   for   detailed
descriptions).

\beginlist%unordered

\atindex{option ClassBound}{@option \noexpand`ClassBound'}
\item{}`ClassBound := <n>'

\atindex{option Exponent}{@option \noexpand`Exponent'}
\item{}`Exponent := <n>'

\atindex{option Metabelian}{@option \noexpand`Metabelian'}
\item{}`Metabelian'

\atindex{option GroupName}{@option \noexpand`GroupName'}
\item{}`GroupName := <name>'

\atindex{option OutputLevel}{@option \noexpand`OutputLevel'}
\item{}`OutputLevel := <n>'

\atindex{option StandardPresentationFile}%
{@option \noexpand`StandardPresentationFile'}
\item{}`StandardPresentationFile := <filename>'

\atindex{option SetupFile}{@option \noexpand`SetupFile'}
\item{}`SetupFile := <filename>'

\atindex{option PqWorkspace}{@option \noexpand`PqWorkspace'}
\item{}`PqWorkspace := <workspace>'

\endlist

The options for `PqStandardPresentation' may also be passed  in  the  two
alternative ways described for  `Pq'  (see~"Pq").  `StandardPresentation'
does not provide these alternative ways of passing options. Options  that
are not passed are set to their default values.

We illustrate the method with the following examples.

\beginexample
gap> F := FreeGroup( "a", "b" );; a := F.1;; b := F.2;;
gap> G := F / [a^25, Comm(Comm(b, a), a), b^5];
<fp group on the generators [ a, b ]>
gap> S := StandardPresentation( G, 5 : ClassBound := 10 );
<fp group on the generators [ f1, f2, f3, f4, f5, f6, f7, f8, f9, f10, f11, 
  f12, f13, f14, f15, f16, f17, f18, f19, f20, f21, f22, f23, f24, f25, f26 ]>
gap> IsPcGroup( S );
false
gap> # if we need to compute with S we should convert it to a pc group
gap> Spc := PcGroupFpGroup( S );
<pc group of size 1490116119384765625 with 26 generators>

gap> H := F / [ a^625, Comm(Comm(Comm(Comm(b, a), a), a), a)/Comm(b, a)^5,
>               Comm(Comm(b, a), b), b^625 ];;                     
gap> StandardPresentation( H, 5 : ClassBound := 15, Metabelian );
<fp group on the generators [ f1, f2, f3, f4, f5, f6, f7, f8, f9, f10, f11, 
  f12, f13, f14, f15, f16, f17, f18, f19, f20 ]>

gap> F4 := FreeGroup( "a", "b", "c", "d" );;                        
gap> a := F4.1;; b := F4.2;; c := F4.3;; d := F4.4;;
gap> G4 := F4 / [ b^4, b^2 / Comm(Comm (b, a), a), d^16,                
>                 a^16 / (c * d), b^8 / (d * c^4) ];
<fp group on the generators [ a, b, c, d ]>
gap> K := Pq( G4 : Prime := 2, ClassBound := 1 );
<pc group of size 4 with 2 generators>
gap> StandardPresentation( G4, K : ClassBound := 14 );
<fp group with 53 generators>
\endexample

Typing: `PqExample( "StandardPresentation" );' runs the above example  in
{\GAP} (see~"PqExample").

(These functions require the {\ANUPQ}  package;  see~"Loading  the  ANUPQ
Package".)

\>EpimorphismPqStandardPresentation( <F>, <p> : <options> ) F
\>EpimorphismPqStandardPresentation( <F>, <Q> : <options> ) F
\>EpimorphismStandardPresentation( <F>, <p> : <options> ) M
\>EpimorphismStandardPresentation( <F>, <Q> : <options> ) M

Each of the above functions accepts the same arguments and options as the
function `StandardPresentation' (see~"StandardPresentation") and  returns
an epimorphism from the fp or pc group <F> onto  the  finitely  presented
group given by a standard  presentation,  i.e.~if  <S>  is  the  standard
presentation    computed    for    the    $p$-quotient    of    <F>    by
`StandardPresentation' then `EpimorphismStandardPresentation' returns the
epimorphism from <F> to the group with presentation <S>.

*Note:*
The  attributes  and  property  `NuclearRank',  `MultiplicatorRank'   and
`IsCapable' are set for the image group of the  epimorphism  returned  by
`EpimorphismPqStandardPresentation' or  `EpimorphismStandardPresentation'
(see Section~"Attributes and a Property for fp and pc p-groups").

We illustrate the function with the following example.

\beginexample
gap> F := FreeGroup(6, "F");
<free group on the generators [ F1, F2, F3, F4, F5, F6 ]>
gap> # For printing GAP uses the symbols F1, ... for the generators of F
gap> x := F.1;; y := F.2;; z := F.3;; w := F.4;; a := F.5;; b := F.6;;
gap> R := [x^3 / w, y^3 / w * a^2 * b^2, w^3 / b,
>          Comm (y, x) / z, Comm (z, x), Comm (z, y) / a, z^3 ];;
gap> Q := F / R;
<fp group on the generators [ F1, F2, F3, F4, F5, F6 ]>
gap> # For printing GAP also uses the symbols F1, ... for the generators of Q
gap> # (the same as used for F) ... but the gen'rs of Q and F are different:
gap> GeneratorsOfGroup(F) = GeneratorsOfGroup(Q);
false
gap> G := Pq( Q : Prime := 3, ClassBound := 3 );
<pc group of size 729 with 6 generators>
gap> phi := EpimorphismStandardPresentation( Q, 3 : ClassBound := 3 );
[ F1, F2, F3, F4, F5, F6 ] -> [ f1*f2^2*f3, f1*f2*f3*f4*f5^2*f6^2, f3^2, f4, 
  f5, f6 ]
gap> NamesOfComponents(phi);
[ "Source", "Range", "MappingGeneratorsImages" ]
gap> phi!.Source; # This is the group Q (GAP uses F1, ... for gen'r symbols)
<fp group on the generators [ F1, F2, F3, F4, F5, F6 ]>
gap> phi!.Range;  # This is the group G (GAP uses f1, ... for gen'r symbols)
<fp group on the generators [ f1, f2, f3, f4, f5, f6 ]>
gap> AssignGeneratorVariables(G); # so f1, ... are now variables
gap> # Just to see that the images of [F1, ..., F6] do generate G
gap> Group([ f1*f2^2*f3, f1*f2*f3*f4*f5^2*f6^2, f3^2, f4, f5, f6 ]) = G;
true
gap> Size( Image(phi) );
729
\endexample

Typing: `PqExample( "EpimorphismStandardPresentation" );' runs the  above
example in {\GAP} (see~"PqExample").

(These functions require the {\ANUPQ}  package;  see~"Loading  the  ANUPQ
Package".)

%%%%%%%%%%%%%%%%%%%%%%%%%%%%%%%%%%%%%%%%%%%%%%%%%%%%%%%%%%%%%%%%%%%%%%%%%
\Section{Testing p-Groups for Isomorphism}

\>IsPqIsomorphicPGroup( <G>, <H> ) F
\>IsIsomorphicPGroup( <G>, <H> ) M

each return true if <G> is isomorphic to <H>, where both <G> and <H> must
be pc groups of prime power order. These functions  compute  and  compare
in {\GAP} the fp groups given by standard presentations for <G>  and  <H>
(see "StandardPresentation").

\beginexample
gap> G := Group( (1,2,3,4), (1,3) );
Group([ (1,2,3,4), (1,3) ])
gap> P1 := Image( IsomorphismPcGroup( G ) );
Group([ f1, f2, f3 ])
gap> P2 := SmallGroup( 8, 5 );
<pc group of size 8 with 3 generators>
gap> IsIsomorphicPGroup( P1, P2 );
false
gap> P3 := SmallGroup( 8, 4 );
<pc group of size 8 with 3 generators>
gap> IsIsomorphicPGroup( P1, P3 );
false
gap> P4 := SmallGroup( 8, 3 );
<pc group of size 8 with 3 generators>
gap> IsIsomorphicPGroup( P1, P4 );
true
\endexample

Typing: `PqExample( "IsIsomorphicPGroup" );' runs the  above  example  in
{\GAP} (see~"PqExample").

(These functions require the {\ANUPQ}  package;  see~"Loading  the  ANUPQ
Package".)

%%%%%%%%%%%%%%%%%%%%%%%%%%%%%%%%%%%%%%%%%%%%%%%%%%%%%%%%%%%%%%%%%%%%%%%%%
\Section{Computing Descendants of a p-Group}

\>PqDescendants( <G> : <options> ) F

returns, for the pc group <G> which must be of prime power order  with  a
confluent pc presentation (see~"ref:IsConfluent!for  pc  groups"  in  the
{\GAP} Reference Manual), a list  of  descendants  (pc  groups)  of  <G>.
Following the colon <options> a selection of  the  options  listed  below
should  be  given,  separated  by  commas  like  record  components  (see
Section~"ref:function call with options" in the {\GAP} Reference Manual).
See Chapter~"ANUPQ options" for detailed descriptions of the options.

\beginlist%unordered

\atindex{option ClassBound}{@option \noexpand`ClassBound'}
\item{}`ClassBound := <n>'

\atindex{option Relators}{@option \noexpand`Relators'}
\item{}`Relators := <rels>'

\atindex{option OrderBound}{@option \noexpand`OrderBound'}
\item{}`OrderBound := <n>'

\atindex{option StepSize}{@option \noexpand`StepSize'}
\item{}`StepSize := <nOrlist>'

\atindex{option RankInitialSegmentSubgroups}%
{@option \noexpand`RankInitialSegmentSubgroups'}
\item{}`RankInitialSegmentSubgroups := <n>'

\atindex{option SpaceEfficient}{@option \noexpand`SpaceEfficient'}
\item{}`SpaceEfficient'

\atindex{option CapableDescendants}{@option \noexpand`CapableDescendants'}
\item{}`CapableDescendants'

\atindex{option AllDescendants}{@option \noexpand`AllDescendants'}
\item{}`AllDescendants := false'

\atindex{option Exponent}{@option \noexpand`Exponent'}
\item{}`Exponent := <n>'

\atindex{option Metabelian}{@option \noexpand`Metabelian'}
\item{}`Metabelian'

\atindex{option GroupName}{@option \noexpand`GroupName'}
\item{}`GroupName := <name>'

\atindex{option SubList}{@option \noexpand`SubList'}
\item{}`SubList := <sub>'

\atindex{option BasicAlgorithm}{@option \noexpand`BasicAlgorithm'}
\item{}`BasicAlgorithm'

\atindex{option CustomiseOutput}{@option \noexpand`CustomiseOutput'}
\item{}`CustomiseOutput := <rec>'

\atindex{option SetupFile}{@option \noexpand`SetupFile'}
\item{}`SetupFile := <filename>'

\atindex{option PqWorkspace}{@option \noexpand`PqWorkspace'}
\item{}`PqWorkspace := <workspace>'

\endlist

*Notes:*
The function `PqDescendants' uses the automorphism group of <G> which  it
computes via the package {\AutPGrp}. If this package is not installed  an
error may be raised. If the automorphism group of <G>  is  insoluble  the
`pq' binary will call {\GAP} together with  the  {\AutPGrp}  package  for
certain orbit-stabilizer calculations.  (So,  in  any  case,  one  should
ensure the {\AutPGrp} package is installed.)

The  attributes  and  property  `NuclearRank',  `MultiplicatorRank'   and
`IsCapable'  are  set  for  each  group   of   the   list   returned   by
`PqDescendants' (see Section~"Attributes and a Property  for  fp  and  pc
p-groups").

The options <options> for `PqDescendants' may be passed in an alternative
manner to that already described, namely you can pass  `PqDescendants'  a
record as an argument, which contains as entries some  (or  all)  of  the
above mentioned. Those parameters which do not occur in  the  record  are
set to their default values.

Note that you cannot set both `OrderBound' and `StepSize'.

In the first example  we  compute all descendants of the Klein four group
which have exponent-2 class at most 5 and order at most $2^6$.

\beginexample
gap> F := FreeGroup( "a", "b" );; a := F.1;; b := F.2;;         
gap> G := PcGroupFpGroup( F / [ a^2, b^2, Comm(b, a) ] );
<pc group of size 4 with 2 generators>
gap> des := PqDescendants( G : OrderBound := 6, ClassBound := 5 );;
gap> Length(des);
83
gap> List(des, Size); 
[ 8, 8, 8, 16, 16, 16, 32, 16, 16, 16, 16, 16, 32, 32, 64, 64, 32, 32, 32, 
  32, 32, 32, 32, 64, 64, 64, 64, 64, 64, 64, 64, 64, 64, 64, 32, 32, 32, 32, 
  64, 64, 64, 64, 64, 64, 64, 64, 64, 64, 64, 32, 32, 32, 32, 32, 64, 64, 64, 
  64, 64, 64, 64, 64, 64, 64, 64, 64, 64, 64, 64, 64, 64, 64, 64, 64, 64, 64, 
  64, 64, 64, 64, 64, 64, 64 ]
gap> List(des, d -> Length( PCentralSeries( d, 2 ) ) - 1 );
[ 2, 2, 2, 2, 2, 2, 2, 3, 3, 3, 3, 3, 3, 3, 3, 3, 3, 3, 3, 3, 3, 3, 3, 3, 3, 
  3, 3, 3, 3, 3, 3, 3, 3, 3, 3, 3, 3, 3, 3, 3, 3, 3, 3, 3, 3, 3, 3, 3, 3, 4, 
  4, 4, 4, 4, 4, 4, 4, 4, 4, 4, 4, 4, 4, 4, 4, 4, 4, 4, 4, 4, 4, 4, 4, 4, 4, 
  4, 4, 4, 5, 5, 5, 5, 5 ]
\endexample

In the second example we compute all  capable descendants of order  27 of
the  elementary abelian group of order 9.  

\beginexample
gap> F := FreeGroup( 2, "g" );
<free group on the generators [ g1, g2 ]>
gap> G := PcGroupFpGroup( F / [ F.1^3, F.2^3, Comm(F.1, F.2) ] );
<pc group of size 9 with 2 generators>
gap> des := PqDescendants( G : OrderBound := 3, ClassBound := 2,
>                              CapableDescendants );
[ <pc group of size 27 with 3 generators>, 
  <pc group of size 27 with 3 generators> ]
gap> List(des, d -> Length( PCentralSeries( d, 3 ) ) - 1 );
[ 2, 2 ]
gap> # For comparison let us now compute all descendants
gap> PqDescendants( G : OrderBound := 3, ClassBound := 2);
[ <pc group of size 27 with 3 generators>, 
  <pc group of size 27 with 3 generators>, 
  <pc group of size 27 with 3 generators> ]
\endexample

In  the  third  example,  we  compute  all  capable  descendants  of  the
elementary abelian group of order  $5^2$ which have exponent-$5$ class at
most $3$, exponent $5$, and are metabelian.

\beginexample
gap> F := FreeGroup( 2, "g" );;                                  
gap> G := PcGroupFpGroup( F / [ F.1^5, F.2^5, Comm(F.2, F.1) ] );
<pc group of size 25 with 2 generators>
gap> des := PqDescendants( G : Metabelian, ClassBound := 3,
>                              Exponent := 5, CapableDescendants );
[ <pc group of size 125 with 3 generators>, 
  <pc group of size 625 with 4 generators>, 
  <pc group of size 3125 with 5 generators> ]
gap> List(des, d -> Length( PCentralSeries( d, 5 ) ) - 1 );
[ 2, 3, 3 ]
gap> List(des, d -> Length( DerivedSeries( d ) ) );
[ 3, 3, 3 ]
gap> List(des, d -> Maximum( List( Elements(d), Order ) ) );     
[ 5, 5, 5 ]
\endexample

The     examples     `"PqDescendants-1"',     `"PqDescendants-2"'     and
`"PqDescendants-3"' (in order) are essentially  the  same  as  the  above
three examples (see~"PqExample").

(The function requires  the  {\ANUPQ}  package;  see~"Loading  the  ANUPQ
Package".)

\>PqList( <filename> [: SubList := <sub> ]) F

reads a file with name <filename> (a string) and returns the list <L>  of
pc groups (or with option `SubList' a sublist of <L> or a single pc group
in <L>) defined in that file. If the option `SubList' is passed  and  has
the value <sub>, then it has the same  meaning  as  for  `PqDescendants',
i.e.~if  <sub>  is  an  integer  then  `PqList'   returns   `<L>[<sub>]';
otherwise, if <sub> is a list of integers `PqList' returns  `Sublist(<L>,
<sub> )'.

Both `PqList' and `SavePqList' (see "SavePqList") can be used to save and
restore a list of descendants (see "PqDescendants").

(The function requires  the  {\ANUPQ}  package;  see~"Loading  the  ANUPQ
Package".)

\>SavePqList( <filename>, <list> ) F

writes a list of descendants <list> to a file  with  name  <filename>  (a
string).

`SavePqList' and `PqList' (see "PqList") can be used to save and restore,
respectively, the results of `PqDescendants' (see "PqDescendants").

(The function requires  the  {\ANUPQ}  package;  see~"Loading  the  ANUPQ
Package".)

%%%%%%%%%%%%%%%%%%%%%%%%%%%%%%%%%%%%%%%%%%%%%%%%%%%%%%%%%%%%%%%%%%%%%%%%%
%%
%E
